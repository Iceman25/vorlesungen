\chapter{Financial Management}

\section{Einführung}

\subsection{Was ist Financial Management}
Financial Management ist die zielgerichtete Beschaffung, Verwendung und Steuerung von unternehmerischem Kapital.


\subsection{Rahmenbedingungen und Ziele des Financial Management}
\begin{itemize}
	\item Vermeidung von Illiquidität und Insolvenz (dauerhafte Illiquidität oder Überschuldung)
	\item Maximierung des investierten Kapitals und des Unternehmenswertes
\end{itemize}


\subsection{Grundlagen: Der Kapitalbedarf eines Unternehmens}
\begin{itemize}
	\item Finanzielles Gleichgewicht: Es muss zu jedem Zeitpunkt möglich sein, dass ein Unternehmen seinen unaufschiebbaren Zahlungsverpflichtungen nachkommt
	\item \textbf{Ermittlung des Zahlungsmittelbestands}
	\begin{enumerate}
		\item Zahlungsmittelanfangsbestand
		\item Cash Flow aus operativer Tätigkeit
		\item Cash Flow aus Investitionstätigkeit
		\item Cash Flow aus Finanzierungstätigkeit
		\item Zahlungsmittelendbestand
	\end{enumerate}
	\item \textbf{Planung des Kapitalbedarfs}
	\begin{itemize}
		\item Liquiditätsplan: Operatives Geschäft, kurzfristige Planung für maximal ein Jahr
		\item Invistitionsplan: Mittel- bis langfristige Finanzplanung, z.B. Neuanschaffungen oder Markterschließungen
	\end{itemize}
\end{itemize}


\subsection{Grundlagen: Formen der Investition/Finanzierung}
\begin{itemize}
	\item \textbf{Außenfinanzierung}
	\begin{itemize}
		\item Fremdkapital (Kredite, Anleihen)
		\item Eigenkapital (Aktien, Gesellschafteranteile)
	\end{itemize}
	\item \textbf{Innenfinanzierung}
	\begin{itemize}
		\item Selbstfinanzierung (Einbehaltene Gewinne, Rücklagen)
		\item Sonstige Finanzierung (Rückstellungen, Desinvestitionen)
	\end{itemize}
\end{itemize}


\subsection{Diskussion: Sharholder Value vs. Stakeholder Value}
\begin{itemize}
	\item Shareholder Value: Ausrichtung der unternehmerischen Tätigkeit an monitären Interessen der Kapitalgeber
	\item Stakeholder Value: Fokusierung auf unterschiedliche Zielgruppen, Mitberücksichtigung gesellschaftlicher Verantwortung, Identifikation der wichtigen Stakeholder notwendig
\end{itemize}



\section{Kurzfristfinanzierung und Working Capital Management}

\subsection{Cash Management}

\subsubsection{Zahlungsmittel und Zahlungsmitteläquivalente}
\begin{itemize}
	\item Kassenbestand
	\item Schecks und Guthaben bei Kreditinstituten
	\item Finanztitel mit einer Restlaufzeit von weniger als drei Monaten (d.h. mit geringen Wertschwankungsrisiken)
\end{itemize}

\subsubsection{Liquiditätshaltung}
\begin{itemize}
	\item \textbf{Motive}
	\begin{itemize}
		\item Vorsichtsmotiv und strategische Motive
		\item Transaktionsmotive (Ein- und Auszahlungen sind i.d.R. nicht synchron)
	\end{itemize}
	\item \textbf{Kosten der Liquiditätshaltung}
	\begin{itemize}
		\item Oppertunitätskosten (Entgangene Zinserträge, Verzicht auf Ausschüttung)
		\item Transaktionskosten (Verkauf von langfristigen Vermögensgegenständen, Kosten kurzfristiger Kreditaufnahme)
	\end{itemize}
\end{itemize}


\subsection{Messung der Liquidität}

\subsubsection{Betriebliche Kennzahlen}
\begin{itemize}
	\item Cash Ratio:
\end{itemize}


\subsection{Working Capital Management}


\subsection{Formen der Kurzfristfinanzierung}




\section{Appendix A: Formelsammlung}

\subsection{Betriebliche Kennzahlen}

\subsubsection{Cash Ratio}
\[Cash~Ratio = \frac{liquide~Mittel}{kurzfristige~Verbindlichkeiten}\]
