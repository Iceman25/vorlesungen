\chapter{Financial Management}

Zusammenfassung der Vorlesung "`Financial Management"' von Professor Ruckes aus dem Sommersemester 2014.\footnote{\url{http://finance.fbv.kit.edu/1505_1675.php}}

\section{Einführung}

\subsection{Was ist Financial Management}
Financial Management ist die zielgerichtete Beschaffung, Verwendung und Steuerung von unternehmerischem Kapital.


\subsection{Rahmenbedingungen und Ziele des Financial Management}
\begin{itemize}
	\item Vermeidung von Illiquidität und Insolvenz (dauerhafte Illiquidität oder Überschuldung)
	\item Maximierung des investierten Kapitals und des Unternehmenswertes
\end{itemize}


\subsection{Grundlagen: Der Kapitalbedarf eines Unternehmens}
\begin{itemize}
	\item Finanzielles Gleichgewicht: Es muss zu jedem Zeitpunkt möglich sein, dass ein Unternehmen seinen unaufschiebbaren Zahlungsverpflichtungen nachkommt
	\item \textbf{Ermittlung des Zahlungsmittelbestands}
	\begin{enumerate}
		\item Zahlungsmittelanfangsbestand
		\item Cash Flow aus operativer Tätigkeit
		\item Cash Flow aus Investitionstätigkeit
		\item Cash Flow aus Finanzierungstätigkeit
		\item Zahlungsmittelendbestand
	\end{enumerate}
	\item \textbf{Planung des Kapitalbedarfs}
	\begin{itemize}
		\item Liquiditätsplan: Operatives Geschäft, kurzfristige Planung für maximal ein Jahr
		\item Invistitionsplan: Mittel- bis langfristige Finanzplanung, z.B. Neuanschaffungen oder Markterschließungen
	\end{itemize}
\end{itemize}


\subsection{Grundlagen: Formen der Investition/Finanzierung}
\begin{itemize}
	\item \textbf{Außenfinanzierung}
	\begin{itemize}
		\item Fremdkapital (Kredite, Anleihen)
		\item Eigenkapital (Aktien, Gesellschafteranteile)
	\end{itemize}
	\item \textbf{Innenfinanzierung}
	\begin{itemize}
		\item Selbstfinanzierung (Einbehaltene Gewinne, Rücklagen)
		\item Sonstige Finanzierung (Rückstellungen, Desinvestitionen)
	\end{itemize}
\end{itemize}


\subsection{Diskussion: Sharholder Value vs. Stakeholder Value}
\begin{itemize}
	\item Shareholder Value: Ausrichtung der unternehmerischen Tätigkeit an monitären Interessen der Kapitalgeber
	\item Stakeholder Value: Fokusierung auf unterschiedliche Zielgruppen, Mitberücksichtigung gesellschaftlicher Verantwortung, Identifikation der wichtigen Stakeholder notwendig
\end{itemize}



\section{Kurzfristfinanzierung und Working Capital Management}

\subsection{Cash Management}

\subsubsection{Zahlungsmittel und Zahlungsmitteläquivalente}
\begin{itemize}
	\item Kassenbestand
	\item Schecks und Guthaben bei Kreditinstituten
	\item Finanztitel mit einer Restlaufzeit von weniger als drei Monaten (d.h. mit geringen Wertschwankungsrisiken)
\end{itemize}

\subsubsection{Liquiditätshaltung}
\begin{itemize}
	\item \textbf{Motive}
	\begin{itemize}
		\item Vorsichtsmotiv und strategische Motive
		\item Transaktionsmotive (Ein- und Auszahlungen sind i.d.R. nicht synchron)
	\end{itemize}
	\item \textbf{Kosten der Liquiditätshaltung}
	\begin{itemize}
		\item Oppertunitätskosten (Entgangene Zinserträge, Verzicht auf Ausschüttung)
		\item Transaktionskosten (Verkauf von langfristigen Vermögensgegenständen, Kosten kurzfristiger Kreditaufnahme)
	\end{itemize}
\end{itemize}


\subsection{Messung der Liquidität}

\subsubsection{Betriebliche Kennzahlen}
\begin{itemize}
	\item \textbf{Cash Ratio (Liquidität 1. Grades)}
	\begin{itemize}
		\item Gibt an, in wieweit ein Unternehmen seine derzeitigen Zahlungsverpflichtungen durch seine Liquide Mitteln erfüllen kann
		\item Idee: Je höher das Cash Ratio, desto liquider das Unternehmen
	\end{itemize}
	\item \textbf{Acid Test Ratio (Liquidität 2. Grades)}
	\begin{itemize}
		\item Erweiterung der Cash Ratio um kurzfristige Forderungen
		\item Wenn \(ATR < 1\): ein Teil der kurzfristigen Verbindlichkeiten ist nicht durch kurzfristig zur Verfügung stehendes Vermögen gedeckt
		\item Achtung: Forderungsausfall möglich
	\end{itemize}
	\item \textbf{Current Ratio (Liquidität 3. Grades)}
	\begin{itemize}
		\item Erweiterung des Acid Test Ratios um Vorräte
		\item Als Untergrenze gilt ein Wert > 1, ansonsten muss eventuell zur Deckung der kurzfristigen Verbindlichkeiten Anlagevermögen veräußert werden
	\end{itemize}
\end{itemize}


\subsection{Working Capital Management}
Aus der Kapitalbindung im Produktionsprozess resultiert ein bestimmter Kapitalbedarf. Dieser Kapitalbedarf sollte so gemanagt werden, dass die resultierenden Gesamtkosten minimiert werden.
\begin{itemize}
	\item Non Cash Working Capital: Vermögensteile, die sich innerhalb eines Produktionszyklus in liquide Mittel zurück "`verwandeln"'
	\item Net Working Capital: Ausdruck der Kapitalbindung im Produktionsprozess (Operatives Umlaufvermögen). Die Bedeutung ist stark branchenabhängig
\end{itemize}

\subsubsection{Ziel des Working Capital Management}
Reduzierung des Net Working Capital (kürzere Kapitalbindung) und somit Reduktion der Finanzirungskosten.

\subsubsection{Ansatzpunkte des Working Capital Management}
\begin{itemize}
	\item Management der Verbindlichkeiten
	\item Management der Vorratshaltung und Produktion
	\item Forderungsmanagement
\end{itemize}

\subsubsection{Maßnahmen}
\begin{enumerate}
	\item \textbf{Management der Vorratshaltung}
	\begin{itemize}
		\item Standardisierung von Bauteilen
		\item Beschaffungslagerhaltungsoptimierung (Extremfall: Just-in-Time Produktion)
		\item Optimierung der Durchlaufzeiten
		\item Gute Marktforschung und Absatzplanung reduzieren
		\item Direkter Zusammenhang zwischen leistungswirtschaftlichen Entscheidungen und dem Kapitalbedarf bzw. den Finanzierungskosten
	\end{itemize}
	\item \textbf{Forderungsmanagement}
	\begin{itemize}
		\item Forderungsbestand optimieren
		\item Handelskredite: Form der Kurzfristfinanzierung. Unternehmen nehmen Kredite von Lieferanten auf und gewähren ihren Kunden Kredite. Zeitliche Verschiebung des Mittelfluss
	\end{itemize}
	\item \textbf{Factoring}
	\begin{itemize}
		\item Verkauf von Forderungen an eine Spezialbank (Factor)
		\item Für einen abgrenzbaren Teil an Forderungen odr einen einzelnen Kreditnehmer möglich
	\end{itemize}
\end{enumerate}


\subsection{Formen der Kurzfristfinanzierung}
Der Finanzierungsbedarf eines Unternehmens hängt von der Bemessung des Net Working Capital (relativ zum Umsatz) ab.
\begin{itemize}
	\item Flexible (oder konservative) Bemessung: Hoher Finanzierungsbedarf, da beispielsweise hohe Lagerbestände oder großzügige Kreditvergabe
	\item Restriktive (oder agressive) Bemessung: Niedrigem Finanzierungsbedarf, allerdings potentieller Verlust von Kunden oder Finanzierungsengpässe
	\item Kurzfristiger Finanzierungsgrund: Saisonale oder temporäre Schwankung des Net Working Capital
\end{itemize}



\section{Fremdkapital}

\subsection{Fremdkapitalkosten}
Übliche Berechnungsweise für Rendite von Anleihen bzw. Effektivverzinsung von Krediten.

\subsubsection{Sicheres und unsicheres Fremdkapital}
\begin{itemize}
	\item Sicheres Fremdkapital: Zins ist der Kapitalmarktzinssatz für risikofreie Kapitalüberlassung unf orientiert sich am Zinssatz für risikolose Staatsanleihen
	\item Unsicheres Fremdkapital: Zins ist die geforderte Rendite der Gläubiger zuzüglich Risikoprämie
\end{itemize}


\subsection{Kreditrisiko}
\begin{itemize}
	\item Screening: Kreditwürdigkeitsprüfung vor Kreditvergabe
	\item Monitoring: Laufende Kreditüberwachung
\end{itemize}

\subsubsection{Ziele der Kreditwürdigkeitsprüfung}
\begin{itemize}
	\item Beurteilung der Ausfallwahrscheinlichkeit
	\item Beurteilung des Verlusts, wenn es zum Ausfall kommt
	\item Auswirkungen auf die Kreditvergabeentscheidung, die Kreditkonditionen und die Eigenkapitalunterlegung
\end{itemize}

\subsubsection{Credit Rating}
\begin{itemize}
	\item Unabhängige (?) Meinung zur Kreditwürdigkeit eines Kreditnehmers
	\item Beinflusst die Möglichkeiten von Unternehmen und Staaten, neues Fremdkapital aufzunehmen
	\item Bei schlechtem Rating ist ein Aufpreis zu zahlen
	\item Ca. 90\% des Marktes ist durch die drei großen Rating-Agenturen (Fitch, Moody's, Standard \& Poor's) abgedeckt
	\item \textbf{Vorgehensweise}
	\begin{enumerate}
		\item Definition der Kriterien zur Beurteilung
		\item Aggregation der Werte für die einzelnen Kriterien $\rightarrow$ Score zur Einteilung in diskrete Rating-Klassen
		\item Schätzung des Zusammenhangs zwischen Score und Ausfallwahrscheinlichkeit
		\item Bei Rating im Firmenkundengeschäft: Dominaz der Jahresabschlussanalyse
	\end{enumerate}
\end{itemize}

\subsubsection{Jahresabschlussanalyse}
\begin{itemize}
	\item Analyse des vergangenheitsbezogenen Zahlenwerks, um Aussagen über die zukünftige Zahlungsfähigkeit zu gewinnen
	\item Von Bedeutung für alle Stakeholder
	\item Angaben im Jahresabschluss werden zu geeigneten Kennzahlen verdichtet. Bsp: Liquiditätskennzahlen oder Z-Score
\end{itemize}


\subsection{Leasing}

\subsubsection{Merkmale}
\begin{itemize}
	\item Für einen bestimmten Zeitraum unkündbar
	\item Leasing-Raten sind unabhängig von der Funktionstüchtigkeit des geleasten Gegenstands
	\item Leasing-Geber erwirtschaftet zusätzliche Gewinnmarge
	\item Vertragsgestaltung umfasst u.a. Vertragsart, Vertragslaufzeit, Zahlungsweise, ergänzende Serviceleistungen, Kündigung, Kaufoption am Ende der Laufzeit
\end{itemize}

\subsubsection{Arten von Leasingverträgen}
\begin{itemize}
	\item \textbf{Financial Lease}
	\begin{itemize}
		\item Leasingnehmer trägt wirtschaftliches Risiko für Wertveränderungen und ist für die Instandhaltung verantwortlich
		\item Leasingnehmer aktiviert das Leasinggut in der Bilanz (als Kauf betrachtet)
		\item Zahlungsreihe entspricht der eines Kreditfinanzierten Kaufs
	\end{itemize}
	\item \textbf{Operating Lease}
	\begin{itemize}
		\item Leasinggeber trägt wirtschaftliches Risiko von Wertänderungen und ist für die Instandhaltung verantwortlich
		\item Leasinggeber aktiviert das Leasinggut in der Bilanz, Leasingnehmer verbucht als Aufwand (Mietverhältnis)
	\end{itemize}
\end{itemize}

\subsubsection{Vorteile von Leasing}
\begin{itemize}
	\item Unterschiedliche steuerliche Behandlung von Leasinggeber und Leasingnehmer
	\begin{itemize}
		\item Leasingnehmer ersetzt den Abzug der Abschreibung und Zinsen durch Abzug der Leasingzahlung
		\item Steuervorteil entsteht, wenn Leasing die höheren Abzüge auf die Partei verlagert, die dem höheren Steuersatz unterliegt
	\end{itemize}
	\item Kostenvorteile des Leasinggebers bei Beschaffung, Wartung oder Verwendung
	\item Reduzierte Wiederverkaufskosten
	\item Risikoübertragung
\end{itemize}


\subsection{Debt Tex Shield}
\begin{itemize}
	\item Fremdfinanzierter Steuervorteil
	\item Unternehmen bezahlen Steuern auf ihren Ertrag nach Abzug der Zinszahlungen $\rightarrow$ Zinsaufwendungen mindern die Höhe der Ertragssteuern
\end{itemize}



\section{Eigenkapital}

\subsection{Eigenkapital vs. Fremdkapital}
\begin{tabularx}{\columnwidth}{|X|X|X|l}
	\hline
		\textbf{Kriterium} & \textbf{Eigenkapital} & \textbf{Fremdkapital} \\
	\hline
		Rechtliche Stellung & Eigentümer & Gläubiger \\
		Haftung	& Haftung in voller Höhe, nachrangiger Anspruch im Insolvenzfall & Keine Haftung, vorraniger Anspruch im Insolvenzfall \\
		Zeitliche Verfügbarkeit & unbefristet & befristet \\
		Partizipation & Stimmrecht, Recht auf Geschäftsführung & Kein Recht auf Geschäftsführung \\
		Beteiligung am Unternehmenserfolg & Teilhabe an variablem Gewinn/Verlust & Lediglich fester Zinsanspruch \\
		Steuerliche Behandlung & Ertragssteuer & Steuerliche Entlastung durch Zinszahlung \\
		Belastung der Liquidität & Ausschüttung nicht verpflichtend & Verpflichtende fixe Zinszahlung Tilgung \\
	\hline
\end{tabularx}


\subsection{Venture Capital}

\subsubsection{Private Equity}
\begin{itemize}
	\item Frühphasenfinanzierung bzw. Venture Capital
	\item \textbf{Spätphasenfinanzierung}
	\begin{itemize}
		\item Minderheitsbeteilung
		\item Mehrheitsbeteiligung (Buyouts)
	\end{itemize}
\end{itemize}

\subsubsection{Quellen für Venture Capital}
\begin{itemize}
	\item Business Angles
	\item Venture Capital Funds: Unabhängige, assoziiert mit einer Bank oder Unternehmensberatung, Corporate Venture Capitalists
	\item Öffentlich geförderte, nichtrenditeorientierte Beteiligungsgesellschaften (z.B. KfW)
\end{itemize}

\subsubsection{Eigenschaften}
\begin{itemize}
	\item Zurverfügungstellung von haftendem Eigenkapital
	\item Mehrheitsbeteiligung bzw. nennenswerte Minderheitsbeteiligung
	\item Zeitliche Befristung der Finanzierung (oft drei bis acht Jahre)
	\item "`Strategische Partnerschaft"': Venture Capitalist unterstützt Management durch Beratungsleistung
	\item Zertifizierungsfunktion, die zur Erschließung weiterer Finanzierungsquellen führen kann
	\item Strukturierung, die VC-Gesellschaft Kontrollrechte sichert
\end{itemize}


\subsection{Börsengänge (Initial Public Offerings)}
Aktien eines noch nicht börsennotierendem Unternehmens werden einem breiten Publikum zur Zeichnung angeboten und danach an der Börse gehandelt. Dient somit der Unternehmensfinanzierung, aber auch dem Exit von Investoren und anderen Alteigentümern.

\subsubsection{Herkunft der angebotenen Aktien}
\begin{itemize}
	\item Verkauf durch Altaktionäre
	\item Verkauf von Aktien aus einer Kapitalerhöhung
	\item Häufig: Mixed Offerings
\end{itemize}

\subsubsection{bookbuilding-Verfahren}
\begin{enumerate}
	\item Veröffentlichung einer Preisspanne
	\item Erteilung Zeichnungsaufträge von Investoren
	\item Emissionsbank (Bookrunner) legt endgültigen Preis fest, i.d.R so, dass Überzeichnung resultiert
	\item Emissionsprospekt kann Erhöhung der Spanne während der Zeichnungsfrist erlauben
\end{enumerate}
Seit Mitte der 90er Jahre auch in Deutschland das dominierende Verfahren.

\subsubsection{Kosten eines Börsengangs}
\begin{itemize}
	\item Direkte Kosten: Underwriting Fee, Börsengebühren, Anwaltskosten
	\item Indirekte Kosten: Underpricing
	\item Publizitätskosten in folge des IPD: Investor Relations
\end{itemize}

\subsubsection{Nutzen}
\begin{itemize}
	\item Überwindung von Finanzierungsrestriktionen
	\item Niedrigere Finanzierungskosten, z.B. durch bessere Verhalungsposition wegen zusätzlicher Finanzierungsalternativen
	\item Diversifikation der Portfolios der Alteigentümer
	\item Informationsfunktion der Marktpreise (z.B. erfolgsabhängige Vergütung)
	\item Kontrolltransfer: u.a. Trennung von Eigentum und Management
	\item Ausnutzung von Fehlbewertungen (Timing des Emissionszeitpunkts)
\end{itemize}


\subsection{Kapitalerhöhungen (Seasoned Offerings)}
Kapitalerhöhung durch:
\begin{itemize}
	\item Zuführung von Mitteln durch bisherige Eigentümer (Rights offer)
	\item Zuführung von Mitteln durch neue Eigentümer (Cash offer)
	\item Beides
\end{itemize}

\subsubsection{Bestimmungen}
\begin{itemize}
	\item Ausgabepreis der neuen Aktien darf nicht unter dem (fiktiven) Nennwert der Aktien liegen
	\item Nennwert der Aktie gibt an, mit welchem Anteil ein Aktionär am Grundkapital einer Aktiengesellschaft beteiligt ist (laut Aktiengesetz mind. 1EUR)
\end{itemize}


\subsection{Appendix: Hybride Finanzierungsformen}

\subsubsection{Genussscheine}
\begin{itemize}
	\item Nicht rechtsformspezifische EK-Form
	\item Nicht gesetzlich geregelt $\rightarrow$ großer Gestaltungsspielraum
	\item Nachrangigkeit, Verlustbeteiligung, ergebnisabhängige Vergütung. Etrag (in)direkt an das Jahresergebnis oder die Ausschüttung gekoppelt
	\item Augestaltung möglich, dass Genussscheine ökonomisches Eigenkapital sind, die Zahlungen an die Genussscheininhaber steuerlich jedoch wie Fremdkapitalzinsesn behandelt werden (Debt Tax Shield nutzbar)
\end{itemize}

\subsubsection{Langfristiges nachrangiges Fremdkapital}
\begin{itemize}
	\item Ausgestaltung mit extrem langer (mind. 30 Jahre) oder unendlicher Laufzeit, aber mit Kündigungsrecht des Emittenten
	\item Nachrangigkeit hinter allen Gläubigern
	\item Werden von Rating-Agenturen oft wie Eigenkapital behandelt
\end{itemize}

\subsubsection{Crowdfunding}
\begin{itemize}
	\item Geldgeber meist \textit{Fans privater Natur}, Engagament z.B. mit Zusatzmaterial oder Erwähnung auf dem Cover "`geehrt"'
	\item Gezielte Investition in Startups und Unternehmen von Privatpersonen
	\item Langfristiger finanzieller Rückfluss erwartet
	\item Partiarische Darlehen in kleinen bis mittleren Summen
\end{itemize}



\section{Kapitalstrukturen und Kapitalkosten}

\subsection{Kapitalstrukturrisiko und Leverage-Effekt}
\begin{itemize}
	\item Stärkere Verschuldung erhöht das EK-Risiko
	\item Die Standardabweichung der EK-Rendite steigt linear mit dem Verschuldungsgrad $\rightarrow$ Maß für das Kapitalstrukturrisiko (Finanzierungsrisiko)
\end{itemize}


\subsection{Irrelevanz der Kapitalstruktur: Modigliani-Miller-Thoreme}

\subsubsection{Idealisierte Modellannahmen}
\begin{itemize}
	\item Vollkommener und vollständiger Kapitalmarkt
	\item Rationale Martktteilnehmer: Keine Arbitagemöglichkeit bleibt ungenutzt
	\item Gegebenes und von der Kapitalstruktur unabhängiges Investitionsprogramm des Unternehmens
	\item Der Unternehmenswert ergibt sich aus der Summe von Eigenkapital und Fremdkapital
\end{itemize}

\subsubsection{Die Modigliani-Miller-Theoreme}
\begin{enumerate}
	\item Die Gesamtmarktwerte zweier Unternehmen der gleichen Risikoklasse, die gleiche erwartete Bruttogewinne aufweisen, sind identisch, unabhängig von der Kapitalstruktur
	\item Die Eigenkapitalkosten sind eine lineare Funktion des Verhältnisses der Marktwerte von Fremd- und Eigenkapital. Sie sind also eine lineare Funktion der Kapitalstruktur
	\item Die Gesamtkapitalkosten zweier Unternehmen der gleichen Risikoklasse, die gleiche erwartete Bruttogewinne aufweisen, sind identisch und unabhängig von der Kapitalstruktur. Sie entsprechen den Eigenkapitalkosten eines unverschuldeten Unternehmens
\end{enumerate}

\subsubsection{Separationstheorem}
Die Irrelevanz der Kapitalstruktur und Investitionsentscheidungen.
\begin{itemize}
	\item Grundsätzlich: Die Kapitalkosten eines Unternehmens entsprechen dem richtigen Kalkulationszins bei der Bewertung von Investitionsprojekten
	\item Unabhängigkeit der Kapitalkosten $\rightarrow$ Kalkulationszins ist unabhängig von der Kapitalstruktur
	\item Investitionsentscheidungen können damit unabhängig von Finanzierungsentscheidungen getroffen werden
	\item Aus den Irrelevenztheoremen folgt also das Separationstheorem
\end{itemize}


\subsection{Trade-off Theorie und optimale Kapitalstruktur}

\subsubsection{Berücksichtigung von nicht-neutralen Steuern}
In ihrem Nachfolgeartikel berücksichten MM das DTS, also die steuerliche Ungleichbehandlung von Eigen- und Fremdkapital. Daraus folgt für die optimale Kapitalstruktur, dass reine bzw. ein ausreichend große Fremdfinanzierung die Steuerlast für das Unternehmen auf Null reduziert.

\subsubsection{Berücksichtigung von Insolvenzkosten}
Direkt Insolvenzkosten (Verfahrenskosten: Gerichte, Anwälte, Insolvenzverwalter, etc) und indirekte Insolvenzkosten resultieren darin, dass das Management und/oder die Eigentümer in Krisensituationen Anreize haben, sich auf eine Weise zu verhalten, die den Gläubigern schadet.

\subsubsection{Lösungsansätze Kapitalstruktur}
\begin{itemize}
	\item \textbf{Vorteile von Fremdkapital}
	\begin{itemize}
		\item Höhere Verschuldung reduziert Marktwert des des Eigenkapitals
		\item Dadurch kann ein Manager einen größeren Teil des Eigenkapitals im Unternehmen halten $\rightarrow$ Angleichung der Interessen von Managern und Eigentümern
		\item Höhers Fremdkapital führt zu höheren Auszahlungsverpflichtungen und somit zur Reduktion des Free Cash Flow
	\end{itemize}
	\item \textbf{Nachteile des Fremdkapitals}
	\begin{itemize}
		\item Flexibilitätsverlust durch stärkeren Fokus auf Erhaltung der Liquidität
		\item Asset Substitionen: Ankauf von Vermögenswerten, die risikoreicher sind als diejenigen, die der Kreditnehmer eines Wertpapierkredits nach Ansicht des Kreditgebers zu kaufen beabsichtigte\footnote{\url{http://www.onpulson.de/lexikon/257/asset-substitution/}}
		\item Bei Entzug liquider Mittel droht Unterinvestition
		\item Verzögerte Liquidation: Eigentümer haben einen Anreiz, die Liquidität herauszuzögern (wegen des Optionscharakters des Eigenkapitals: \textit{gsmbling for resuurection})
	\end{itemize}
\end{itemize}


\subsection{Kapitalkosten}

\subsubsection{Weighted Average Cost of Capital}
Der gewichtete durchschnittliche Kapitalkostensatz wird von vielen Unternehmen verwendet, um die Mindestrendite für Investitionsprojekte zu bestimmen.\footnote{\url{http://de.wikipedia.org/wiki/WACC-Ansatz}}
\begin{itemize}
	\item Zur Bestimmung sollten stets Marktwerte genommen werden
	\item Teilweise problematisch bei Unternehmen zu ermitteln, da diese i.d.R. nur einen Teil ihres Fremdkapitals über den Kapitalmarkt beziehen
\end{itemize}

\subsubsection{Fremdkapitalkosten \(r_{FK}\)}
\begin{itemize}
	\item Als Rendite wird die Yield-To-Maturity verwendet (Diskontsatz, der die Zahlungsreihe einer Anleihe auf den heutigen Marktwert bringt)
	\item Hat ein Unternehmen keine ausstehenden Anleihen, verwendet man i.A. den yield spread zu risikolosen Staatsanleihen
\end{itemize}

\subsubsection{Eigenkapitalkosten \(r_{EK}\)}
\begin{itemize}
	\item Verwendung des CAPM
\end{itemize}



\section{Einführung in die neoinstitutionalistische Finanzierungstheorie}

\subsection{Einführung}
Grundgedanke der Neoklassik ist ein reibungslos funktionierender Markt (vollständig und vollkommen) mit gegebenen Preisen für Zahlungsströmen und symmetrischer Informationsverteilung $\rightarrow$ First-Best-Lösung erreichbar.

\subsubsection{Moderne Finanzierungstheorie: Nutzenmaximierung!}
\begin{itemize}
	\item Möglichst geringe Gegenleistung für Preis
	\item Asymmetrische Informationsverteilung
	\item Interessen der Vertragspartner können in Konflikt stehen
	\item Potentialles opportunistisches Verhalten
\end{itemize}

\subsubsection{Interessenskonflikte der Vertragsparteien}
\begin{itemize}
	\item Hidden characteristics: Asymmetrische Informationsverteilung, übervorteilter Marktteilnehmer muss sich aus dem Markt zurückziehen $\rightarrow$ nur ein Bruchteil der Transaktionen findet statt (unvollkommener Markt)
	\item Hidden action: Informationsasymmetrie nach Vertragsabschluss (Bsp.: Reduktion des Arbeitseinsatzes bei fixer Entlohnung von Managern)
	\item Nachverhandlungen: Eine Vertragspartei tätigt nach Vertragsabschluss irreversible Investitionen, die andere Partei nutzt dies für nachträgliche Vertragskorrekturen zu ihren Gunsten
\end{itemize}

\subsubsection{Entgegenwirkungen}
\begin{itemize}
	\item Screening
	\item Monitoring
	\item Optimale, anreizkompatible Vertragsgestaltung
	\item Einrichtung geeigneter (Finanz-)Institutionen (Verträge, Gesetze, etc.)
\end{itemize}
Verursachen allerdings Transaktionskosten.


\subsection{Adverse Selection - Die "`Pecking Order"' Theorie}
Die Peckign Order Theorie besagt, dass Unternehmen ihre Finanzierungsquellen (von interner Finanzierung zu Eigenkapital) nach dem Prinzip des geringsten Aufwands oder des geringsten Widerstandes priorisieren, und es vorziehen Eigenkapital als Finanzierungsquelle lediglich als letzten Ausweg heranzuziehen. Daher werden interne Mittel (wie Gewinnthesaurierung) zuerst herangezogen, wenn diese aufgebraucht sind, werden Schulden aufgenommen und wenn es nicht mehr sinnvoll ist noch mehr Schulden aufzunehmen, wird Eigenkapital ausgegeben.\footnote{\url{http://de.wikipedia.org/wiki/Hackordnungstheorie}}


\subsection{Moral Hazard bei Fremdkapital}
Moral Hazard Problematik zwischen Eigentümern und Fremdkapitalgebern: Der Wert des Eigenkapitals steigt, wenn die Volatilität der Cash-Flows erhöht wird. Daher haben Eigenkapitalgeber ein Interesse daran, nachdem die Verträge mit den Fremdkapitalgebern abgeschlossen sind, das Risiko der Investitionsprojekte zum Nachteil der Fremdkapitalgeber zu erhöhen $\rightarrow$ Asset-Substitution Problem.


\subsection{Unterinvestition und Nachverhandlung}
Neue Kapitalgeber sind nicht bereit, ein an sich vorteilhaftes Projekt zu finanzieren, weil die Erträge in erster Linie den alten Gläubigern zu Gute kommen (entsprechendes gilt für Eigenkapital) $\rightarrow$ Unterinvestition.

\subsubsection{Nachverhandlungen}
Nachverhandlungen sind eventuell ex-post-effizient, denn ohne Nachverhandlung würde das Unternehmen liquidiert und das effiziente Projekt würde nicht durchgeführt werden.


\subsection{Fazit}
\begin{itemize}
	\item Vei Vorliegen asymmetrischer Informationsverteilung treten Hidden-Information und Hidden-Action-Probleme auf
	\item Infolgedessen hat die Art der Finanzierung Auswirkungen auf Investitionsentscheidungen $\rightarrow$ die MM-Annahmen sind nicht aufrechtzuerhalten, beispielsweise ist die Kapitalstruktur und die Ausschüttungspolitik \textit{nicht} irrelevant
	\item Institutionelle Details beeinflussen Anreize für Eigentümer und Manager und werden daher wichtig
\end{itemize}



\section{Ausschüttungspolitik}

\subsection{Verwendungen für freie Cash Flows}
\begin{itemize}
	\item \textbf{Thesaurieren}
	\begin{itemize}
		\item Investition in neue Projekte
		\item Erhöhung der Barreserven
	\end{itemize}
	\item \textbf{Auszahlen}
	\begin{itemize}
		\item Aktienrückkauf
		\item Dividendenausschüttung
	\end{itemize}
\end{itemize}


\subsection{Einführung}
\begin{itemize}
	\item Der freie Cash Flow eines Unternehmens bestimmt die Höhe der Ausschüttungen
	\item In einem vollkommenen Kapitalmarkt spielt die Art der Ausschüttung keine Rolle
	\item Wie bei der Kapitalstruktur sind es die Unzulänglichkeiten des Marktes, die die Ausschüttungsstrategie festlegen sollten
\end{itemize}


\subsection{Dividenden}
\begin{itemize}
	\item Können von Unternehmen nicht von der Steuer abgesetzt werden
	\item Sind keine Zahlungsverpflichtungen, können nicht zur Insolvenz führen
	\item Das Datum zwei Tage vor dem Dividendenstichtag wird als Ex-Dividende-Datum bezeichnet
	\item Am Tag der Ausschüttung werden Dividendenschecks an alle registrierten Aktionäre verschickt
\end{itemize}

\subsubsection{Dividend Smoothing}
Langfristig möglichst stabile Dividende gewünscht (Signalwirkung).


\subsection{Aktienrückkäufe}
\begin{itemize}
	\item Alternative, Barmittel an Investoren zu Zahlen, ist der Aktienrückkauf
	\item Unternehmen verwendet Barmittel, um eigene, im Umlauf befindliche Aktien zurückzukaufen
	\item \textbf{Möglichkeiten des Rückkaufs}
	\begin{itemize}
		\item Open Market Purchase: Angekündigt, meist über den Zeitraum eines Jahre, nicht alle annocierten Aktien müssen zurückgekauft werden, häufig verwendet
		\item Tender Offer: Fixes Übernahmeangebot mit Preispremium binnen einer speziellen Frist (etwa 20 Tage), geeignet für große Aktienpakete
	\end{itemize}
	\item Kann steuerlich vorteilhaft sein, wenn realisierte Kursgewinne niedriger besteuert werden als Dividenden
	\item Müssen von der Hauptversammlung genehmigt werden
	\item In Deutschland maximal 10\% rückkaufbar
\end{itemize}


\subsection{Dividenden vs. Aktienrückkäufe}
Investoren haben i.d.R. steuerliche Präferenzen: Es existieren Unterschiede in de Besteuerung von Kapitalausgaben und Dividenden bei unterschiedlichen Investorengruppen (z.B. Pensionskassen).

Dividenden sind in den meisten Ländern mit einenem höheren Steuersatz belegt als Kapitalgewinne, die ein Investor durch den Verkauf von im Kurs gestiegener Aktien erzielen kann.

Trotz ihrer steuerlichen Nachteile bleiben Dividenden ein weiterhin häufig verwendetes Mittel der Ausschüttungspolitik.


\subsection{Aktiensplits}
\begin{itemize}
	\item Vergrößerung der Anzahl an Aktien
	\item Proportionale Veränderung des Aktienkurses
	\item Keine Änderung des Gesamtwertes
	\item Vergleichbar: Ausgabe von Gratisaktien (stock dividend)
\end{itemize}



\section{Aspekte der Investition/Desinvestition und Diversifikation}

\subsection{Überblick}
\begin{itemize}
	\item Um zu diversifizieren, müssen Unternehmen Investitionen tätigen
	\item Um zu refokussieren, müssen Unternehmen Desinvestitionen tätigen, d.h. es werden Unternehmensteile verkauft (Form der Innenfinanzierung)
\end{itemize}


\subsection{Diversifikation}
Eine Produktdiversifikation liegt vor, sofern ein Unternehmen ein neues Produkt auf einem neuen Markt einführt. Das Gegenteil ist die Marktdurchdringung. Die Diversifikation ist der riskanteste Bestandteil der Produkt-Markt-Matrix nach Ansoff. Ein mögliches Maß zur Messung der Diversifikation ist der Berry-Index.\footnote{\url{http://de.wikipedia.org/wiki/Diversifikation_\%28Wirtschaft\%29}}

\subsubsection{Entstehung von Diversifikation}
\begin{itemize}
	\item Intern: Das Unternehmen wächst aus eigener Kraft und entwickelt das Produkt selbst
	\item Übernahme: Ein anderes Unternehmen wird mit dem gewünschten Produkt hinzugekauft
	\item Kooperation: Neue Produkte werden mit einem Partner entwickelt
\end{itemize}

\subsubsection{Berry-Index}
Der Berry-Index gibt den Grad der Diversifikation eines Konzerns an. Berechnet wird er als Komplement der Summe der quadrierten Umsatzanteile (in \%) aller eigenständigen Bereiche eines Unternehmens.\footnote{\url{http://de.wikipedia.org/wiki/Berry-Index}}

\subsubsection{Diversifikation aus unterschiedlichen Sichtweisen}
\begin{itemize}
	\item Investorsicht: Diversifikation bzw. den gewünschten Grad an Risiko kann der Investor selbst in seinem Portefeuille bestimmen
	\item Unternehmenswachstum: Wachstum in einer Branche ist generell beschränkt
	\item Management: Insolvenzrisiko wird reduziert
\end{itemize}



\section{Appendix A: Formelsammlung}

\subsection{Betriebliche Kennzahlen}

\subsubsection{Cash Ratio}
\[Cash~Ratio = \frac{liquide~Mittel}{kurzfristige~Verbindlichkeiten}\]

\subsubsection{Acid Test Ratio (Quick Ratio)}
\[ATR = \frac{LiMi + kurzfristige~Forderungen}{kurzfristige Verbindlichkeiten}\]

\subsubsection{Current Ratio}
\[Current~Ratio = \frac{Umlaufvermoegen}{kurzfristige~Verbindlichkeiten}\]


\subsection{Working Capital Management}

\subsubsection{Bestimmung der Cash Flows}
\begin{enumerate}
	\item CF aus operativer Tätigkeit: \(Jahresueberschuss\) \\ \(+ zahlungsunwirksame~Aufwendungen + Veraendungerungen~im~NWC\)
	\item CF aus Investitionstätigkeit: Alle Veränderungen im Anlagevermögen
	\item CF aus Finanzierungstätigkeit: Ein- und Auszahlungen von und zu Kreditgebern und Eigenkapitalgebern
\end{enumerate}

\subsubsection{Working Capital}
\[WC = (Umlaufvermoegen - LiMi - kurzfr.~finanz.~Vermoegenswerte)\]

\subsubsection{Net Working Capital}
\[NWC = (Umlaufvermoegen - LiMi - kurzfr.~finanz.~Vermoegenswerte)\]
\[- (kurzfr.~Verbindlichkeiten - kurzfr.~Finanzverbindlichkeiten)\]

\subsubsection{Geldumschlagsdauer (Cash Conversion Cycle)}
\[Geldumschlagsdauer = \varnothing Lagerdauer + \varnothing Inkassoperiode - \varnothing Lieferantenzahlungsziel\]
\[OC = \varnothing Lagerdauer + \varnothing Inkassoperiode\]
\[\varnothing Lagerdauer = \frac{\varnothing Lagerbestand \cdot 360}{Jahresverbrauch} = \frac{\varnothing Lagerbestand \cdot 360}{Umsatzkosten}\]
\[\varnothing Inkassoperiode = \frac{\varnothing Forderungen~aus~LuL \cdot 360}{Umsatz~auf~Ziel}\]
\[\varnothing Lieferantenzahlungsziel = \frac{\varnothing Verbindlichkeiten~aus~LuL \cdot 360}{Umsatzkosten}\]
\[Forderungsbestand = \varnothing Forderungslaufzeit \cdot \varnothing Tagesumsatz\]

\subsubsection{Optimale Höhe der Bargeldorder (Baumol Modell)}
\[C* = \sqrt{\frac{2 \cdot Transaktionsgebuehr \cdot \sum Liquiditaetsbedarf}{r}}\]
\[Opportunitaetskosten(C*) = r \cdot \frac{C*}{2}\]
\[Transaktionskosten(C*) = Transaktionsgebuehr \cdot \frac{\sum Liquiditaetsbedarf}{C*}\]

\subsubsection{Miller-Orr Modell}
\[Z* = L* + \sqrt[3]{\frac{3}{4} \cdot \frac{Kosten_{fix} \cdot \sigma_{Tag}^2}{i_{Tag}}}, i_{Tag} = (1+i)^\frac{1}{360}-1\]
\[H* = 3 \cdot Z* - 2 \cdot L*\]
\[\varnothing Kassenbestand = \frac{4 \cdot Z* - 2 \cdot L*}{3}\]


\subsection{Fremdkapital}

\subsubsection{Darlehen mit Disagio}
\[Betrag = \frac{Nennwert}{1 - Disagio}\]

\subsubsection{Fremdkapitalkosten}
\[E_0 = \sum_{t=1}^{n} \frac{A_t}{(1+i)^t}\]

\subsubsection{Kennzahlen zur Unternehmensbonität}
\[Verschuldungsgrad = \frac{Fremdkapital}{Eigenkapital}\]
\[Return~on~Equity = \frac{Jahresueberschuss}{Eigenkapital}\]
\[Nettoumsatzrendite = \frac{Jahresueberschuss}{Umsatz}\]
\[Return~on~Assets = \frac{EBIT}{Bilanzsumme}\]

\subsubsection{Z-Score}
\[zscore = 3,25 + 6,56X_1 + 3,26X_2 + 6,72X_3 + 1,05X_4\]
\begin{itemize}
	\item \(X_1 = \frac{Net~Working~Capital}{Bilanzsummer}\)
	\item \(X_2 = \frac{Einbehaltene~Gewinne}{Bilanzsumme}\)
	\item \(X_3 = \frac{EBIT}{Bilanzsummer}\)
	\item \(X_4 = \frac{Buchwert~des~EK}{Buchwert~der~Verbindlichkeiten}\)
\end{itemize}

\subsubsection{Debt Tex Shield}
\[dts = Ertragssteuersatz \cdot Zinszahlung\]

\subsubsection{Prüfschema Financial vs. Operating Lease}
Kriterien nacheinander abarbeiten, beim ersten "`ja"' handelt es sich um Financial Lease.
\begin{enumerate}
	\item \(Dauer(Leasingvertrag) < 75\%~oekonomische Nutzungsdauer\)
	\item Kaufoption? Eigenumsübergang? Spezialleasing?
	\item Barwertkriterium: \(BW(Leasing) = LR_1 + LR_1 \cdot (\frac{1}{r} - \frac{1}{r \cdot (1+r)^n}), \frac{BW(Leasing)}{Marktwert} > 90\%\) 
\end{enumerate}


\subsection{Eigenkapital - Börsengang}

\subsubsection{Zugeflossenes Kapital}
\[n_{Aktien} \cdot Emissionspreis \cdot (1- Gross~Preceeds)\]

\subsubsection{Marktwert des Unternehmens}
\[Marktwert_{EK} + Marktwert_{FK}\]
\[Marktwert_{EK} = n_{Aktien} \cdot Kurs\]

\subsubsection{Underpricing}
\begin{itemize}
	\item \textit{ex-ante}: Differenz zwischen erwartetem Börsenkurs und Emissionskurs
	\item \textit{ex-post}: Tatsächlich realisierte Differenz zwischen erstem Börsenkurs und Emissionskurs
\end{itemize}

\subsubsection{Rendite nach erstem Handelstag}
\[\frac{Underpricing}{Emissionskurs}\]

\subsubsection{Money left on the Table}
\[Underpricing \cdot n_{Aktien}\]

\subsubsection{Kosten des IPO}
Gross Preceeds gehören zu den direkten Kosten.
\[Kosten_{direkt} + money~left~on~the~table\]

\subsubsection{Erwartungswert uninformierte Anleger}
\[\mathbb{E}(G_n) = p \cdot \frac{n_{Aktien}}{n_{Zeichner}} \cdot (V_H - K) + (1-p) \cdot \frac{n_{Aktien}}{p_{uninf} \cdot n_{Zeichner}} \cdot (V_L-K)\]

\subsubsection{Aktienkurs nach Kapitalerhöhung}
\[K = \frac{n_{Aktien~alt} \cdot K_{alt} + n_{neu} \cdot K_{neu}}{n_{alt} + n_{neu}}\]

\subsubsection{Gewinn je Aktie (EPS)}
\[\frac{Jahresueberschuss}{n_{Aktien}}\]

\subsubsection{Kurs-Gewinn-Verhältnis}
\[\frac{Aktienkurs}{Gewinn~je~Aktie}\]

\subsubsection{Kurs aus Marktkapitalisierung}
\[\frac{Marktkapitalisierung}{n_{Aktien}}\]

\subsubsection{Agio}
\[Agio = Kurs - Aktienkurs\]

\subsubsection{Wert Bezugsrecht}
\[B = \frac{K_{alt} \cdot K_{neu}}{\frac{n_{alt}}{n_{neu}}+1}\]


\subsection{Kapitalstruktur und Kapitalkosten}

\subsubsection{Kapitalstrukturrisiko und Leverage-Effekt}
\[r_{EK} = r_{GK} + \frac{FK}{EK} \cdot (r_{GK} - r_{FK})\]

\subsubsection{WACC}
\[r_{wacc} = \frac{EK}{GK} \cdot r_{EK} + \frac{FK}{GK} \cdot r_{FK} \cdot (1-Einkommenssteuersatz)\]

\subsubsection{Eigenkapitalkosten}
\[r_i = r + (r_M - r) \cdot \beta_i\]
\[\beta_i = \frac{cov(r_i, r_M)}{\sigma_M}\]
\[\sigma_{EK} = (1 + \frac{FK}{EK}) \cdot \sigma_{GK}\]

\subsubsection{Bestimmung Fremdkapitalniveau}
\begin{enumerate}
	\item Zinszahlungen
	\item Debt Tax Shield
	\item Barwert Debt Tax Shield
	\item Insolvenzkosten
	\item Barwert der erwarteten Insolvenzkosten
	\item Vorteil aus BW(Debt Tax Shield) und Barwert der erwarteten Insolvenzkosten
\end{enumerate}


\subsection{Ausschüttungspolitik}

\subsubsection{Zukünftiger Unternehmenswert (Ewige Rente)}
\[heutiger~free~Cash~Flow + Barwert(kuenfitge~free~Cash~Flows)\]
\[= CashFlow_{heute} + \frac{CashFlow_{Zukunft}}{r_{EK}}\]

\subsubsection{Cum Dividend}
\[Div_{cum} = Div_{heute} + BW(Div_{Zukunft})\]

\subsubsection{Ex Dividend}
\[Div_{ex} = BW(Div_{Zukunft})\]

\subsubsection{Dividend Discount Model}
\(g\) bezeichnet die durschnittliche Wachstumsrate.
\[Kurs_{0} = \frac{Div_1 \cdot (1+g)}{r_{EK}-g}\]

\subsubsection{Rendite Gesamtkapital}
\[r_{GK} = \frac{Gewinn}{Gesamtkapital}\]


\subsection{Diversifikation}

\subsubsection{Berry Index}
\[D_B = 1 - \sum_{t=1}^{n} p_i^2\]
