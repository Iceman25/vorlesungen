\section{Appendix A: Algorithmen und Erklärungen}

Aufstellung und Erläuterung aller Algorithmen der Vorlesung Vorlesung "`Algorithmen II"' aus dem Wintersemester 2014.\footnote{\url{http://geom.ivd.kit.edu/ws14_algo2.php}}

\subsection{Algorithmus von Ford und Fulkerson}

\begin{algorithm}[H]
	\caption{Ford-Fulkerson}
	\SetKwData{Left}{left}\SetKwData{This}{this}\SetKwData{Up}{up}
	\SetKwFunction{Union}{Union}\SetKwFunction{FindCompress}{FindCompress}
	\SetKwInOut{Input}{input}\SetKwInOut{Output}{output}

	\Input{$F(G,k,q,s)$}
	\Output{Ein maximaler Fluß $f$}
	\BlankLine

	$f \longleftarrow 0$

	\While{Es gibt einen Pfad $q \rightarrow s$ in $G_f$} {
		Erhöhe $f$ über diesem maximal
	}
\end{algorithm}

\subsection{Edmonds-Karp.Algorithmus}
	Erhöht man den Fluß in Ford-Fulkerson imer längs eines kürzesten Pfads (Breitensuche), erhält man den Edmonds-Karp-Algorithmus
	
\subsection{Die Präfluss-Pusch-Methode}
\begin{algorithm}[H]
	\caption{Push}
	\SetKwData{Left}{left}\SetKwData{This}{this}\SetKwData{Up}{up}
	\SetKwFunction{Union}{Union}\SetKwFunction{FindCompress}{FindCompress}
	\SetKwInOut{Input}{input}\SetKwInOut{Output}{output}

	\Input{$x,y$}
	\Output{}
	\BlankLine

	$d \longleftarrow \min{ü(x), k_f(x,y)}$ \newline
	$f(x,y) \longleftarrow f(x,y) +d$ \newline
	$ü(x) \longleftarrow ü(x) -d $ \newline
	$ü(y) \longleftarrow ü(y) +d$ \newline
\end{algorithm}

Push ist nur erlaubt wenn 
	\begin{itemize}
		\item $ü(x) > 0, x \in V \setminus {q,s}$
		\item $(x,y) \in E_f$
		\item $h(x) - h(y) = 1$
	\end{itemize} 


\begin{algorithm}[H]
	\caption{Lifte}
	\SetKwData{Left}{left}\SetKwData{This}{this}\SetKwData{Up}{up}
	\SetKwFunction{Union}{Union}\SetKwFunction{FindCompress}{FindCompress}
	\SetKwInOut{Input}{input}\SetKwInOut{Output}{output}

	\Input{$x$}
	\Output{}
	\BlankLine

	$h(x) := 1+ \min_{(x,y) \in E_f}{h(y}$
\end{algorithm}

Lifte ist nur erlaubt wenn 
	\begin{itemize}
		\item $x \in V \setminus {q,s}$
		\item $ü(x) > 0$
		\item $h(x) \leqslant \min_{(x,y) \in E_f}{h(y)}$
	\end{itemize} 


\begin{algorithm}[H]
	\caption{Präfluss-Pusch}
	\SetKwData{Left}{left}\SetKwData{This}{this}\SetKwData{Up}{up}
	\SetKwFunction{Union}{Union}\SetKwFunction{FindCompress}{FindCompress}
	\SetKwInOut{Input}{input}\SetKwInOut{Output}{output}

	\Input{$F(G,k,q,s$}
	\Output{Maximaler Fluss $f$}
	\BlankLine

	\For{alle $ x,y \in V $ }{
	$ h(x)  \leftarrow \begin{cases}|V| & x=q \\ 0 & \text{sonst}\end{cases}$ \newline
	$ f(x,y)  \leftarrow \begin{cases}k(x,y) & x=q \\ 0 & \text{sonst}\end{cases}$}
	\While{Es gibt erlaubte Push oder Lifte Operationen} {Führe belibeig Push oder Lifte aus}
\end{algorithm}


\subsection{Binäre Zerlegung des Raum}
Rekursive Zerlegung eines Polyeders des \(\mathbb{R}^3\) anhand einiger Polygone. Pro Iterationsschritt wird jeweils ein Polygon bestimmt (bevorzugt eines, das den Polyeder komplett zerlegt). Dieses teilt den Polyeder in einen linken und einen rechten Teilpolyeder, die jeweils weiter zerlegt werden.

\begin{algorithm}[H]
	\caption{BRZ}
	\SetKwData{Left}{left}\SetKwData{This}{this}\SetKwData{Up}{up}
	\SetKwFunction{Union}{Union}\SetKwFunction{FindCompress}{FindCompress}
	\SetKwInOut{Input}{input}\SetKwInOut{Output}{output}

	\Input{Ein Polyeder $P \subset \mathbb{R}^3$; orientierte, planare, disjunkte Polygone $P_1,...,P_n \subset \mathbb{R}^3$}
	\Output{Ein $RBZ$ für $P_1,...,P_n$}
	\BlankLine

	$k \longleftarrow 1$
	\BlankLine
	\tcc{Sortiere Polygonenteile aus, die außerhalb von $P$ liegen (Clipping)}
	\For {$i=1,...,n$} {
		$Q_k \longleftarrow P_1 \cap P$
		\If {$Q_k \neq \emptyset$} {
			$k \longleftarrow k + 1$
		}
	}

	$l \longleftarrow 1$

	\BlankLine
	\tcc{Falls ein Polygon $P$ komplett zerteilt, nehme dieses als Trennelement. Anderenfalls nehme das erste in der Liste}
	\If {$\exists~j: Q_j~zerlegt~P~vollstaendig$} {
		$l \longleftarrow j$
	}

	$Wurzel \longleftarrow Q_l$
	\BlankLine
	\tcc{Teile $P$}
	\If{$k \geq 3$} {
		$Q \longleftarrow P \cap li.~HR~von~Q_l$\newline
		$li. Wurzelteilbaum \longleftarrow BRZ(Q,Q_1,...,Q_{k-1})$\newline
		$Q \longleftarrow P \cap re.~HR~von~Q_l$\newline
		$re. Wurzelteilbaum \longleftarrow BRZ(Q,Q_1,...,Q_{k-1})$
	}

	\BlankLine
	\Return{Wurzel~mit~ihren~Teilbaeumen}

\end{algorithm}
