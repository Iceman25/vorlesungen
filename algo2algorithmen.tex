\section{Appendix A: Algorithmen und Erklärungen}

Aufstellung und Erläuterung aller Algorithmen der Vorlesung Vorlesung "`Algorithmen II"' aus dem Wintersemester 2014.\footnote{\url{http://geom.ivd.kit.edu/ws14_algo2.php}}

\subsection{Algorithmus von Ford und Fulkerson}

\begin{algorithm}[H]
	\caption{Ford-Fulkerson}
	\SetKwData{Left}{left}\SetKwData{This}{this}\SetKwData{Up}{up}
	\SetKwFunction{Union}{Union}\SetKwFunction{FindCompress}{FindCompress}
	\SetKwInOut{Input}{input}\SetKwInOut{Output}{output}

	\Input{$F(G,k,q,s)$}
	\Output{Ein maximaler Fluß $f$}
	\BlankLine

	$f \longleftarrow 0$

	\While{Es gibt einen Pfad $q \rightarrow s$ in $G_f$} {
		Erhöhe $f$ über diesem maximal
	}
\end{algorithm}

\subsection{Edmonds-Karp.Algorithmus}
	Erhöht man den Fluß in Ford-Fulkerson imer längs eines kürzesten Pfads (Breitensuche), erhält man den Edmonds-Karp-Algorithmus
	
\subsection{Die Pfäfluss-Pusch-Methode}
\begin{algorithm}[H]
	\caption{Push}
	\SetKwData{Left}{left}\SetKwData{This}{this}\SetKwData{Up}{up}
	\SetKwFunction{Union}{Union}\SetKwFunction{FindCompress}{FindCompress}
	\SetKwInOut{Input}{input}\SetKwInOut{Output}{output}

	\Input{$x,y$}
	\Output{}
	\BlankLine

	$d \longleftarrow \min{ü(x), k_f(x,y)}$ \newline
	$f(x,y) \longleftarrow f(x,y) +d$ \newline
	$ü(x) \longleftarrow ü(x) -d $ \newline
	$ü(y) \longleftarrow ü(y) +d$ \newline
\end{algorithm}

Push ist nur erlaubt wenn 
	\begin{itemize}
		\item $ü(x) > 0, x \in V \setminus {q,s}$
		\item $(x,y) \in E_f$
		\item $h(x) - h(y) = 1$
	\end{itemize} 


\begin{algorithm}[H]
	\caption{Lifte}
	\SetKwData{Left}{left}\SetKwData{This}{this}\SetKwData{Up}{up}
	\SetKwFunction{Union}{Union}\SetKwFunction{FindCompress}{FindCompress}
	\SetKwInOut{Input}{input}\SetKwInOut{Output}{output}

	\Input{$x$}
	\Output{}
	\BlankLine

	$h(x) := 1+ \min_{(x,y) \in E_f}{h(y}$
\end{algorithm}

Lifte ist nur erlaubt wenn 
	\begin{itemize}
		\item $x \in V \setminus {q,s}$
		\item $ü(x) > 0$
		\item $h(x) \leqslant \min_{(x,y) \in E_f}{h(y)}$
	\end{itemize} 


\begin{algorithm}[H]
	\caption{Präfluss-Pusch}
	\SetKwData{Left}{left}\SetKwData{This}{this}\SetKwData{Up}{up}
	\SetKwFunction{Union}{Union}\SetKwFunction{FindCompress}{FindCompress}
	\SetKwInOut{Input}{input}\SetKwInOut{Output}{output}

	\Input{$F(G,k,q,s$}
	\Output{Maximaler Fluss $f$}
	\BlankLine

	\For{alle $ x,y \in V $ }{
	$ h(x)  \leftarrow \begin{cases}|V| & x=q \\ 0 & \text{sonst}\end{cases}$ \newline
	$ f(x,y)  \leftarrow \begin{cases}k(x,y) & x=q \\ 0 & \text{sonst}\end{cases}$}
	\While{Es gibt erlaubte Push oder Lifte Operationen} {Führe belibeig Push oder Lifte aus}
\end{algorithm}



\subsection{Binäre Zerlegung des Raum}

\begin{algorithm}[H]
	\caption{BRZ}
	\SetKwData{Left}{left}\SetKwData{This}{this}\SetKwData{Up}{up}
	\SetKwFunction{Union}{Union}\SetKwFunction{FindCompress}{FindCompress}
	\SetKwInOut{Input}{input}\SetKwInOut{Output}{output}

	\Input{Ein Polyeder $P \subset \mathbb{R}^3$; orientierte, planare, disjunkte Polygone $P_1,...,P_n \subset \mathbb{R}^3$}
	\Output{Ein $RBZ$ für $P_1,...,P_n$}
	\BlankLine

	$k \longleftarrow 1$

	\For{$i=1,...,n$} {
		$Q_k \longleftarrow P_1 \cap P$\newline
		if {$Q_k \neq \emptyset$} {
			$k \longleftarrow k + 1$
		}
	}

	$l \longleftarrow 1$

	if {$\exists~j: Q_j zerlegt~P~vollstaendig$} {
		$l \longleftarrow j$
	}
\end{algorithm}
