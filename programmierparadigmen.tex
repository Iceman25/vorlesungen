\chapter{Programmierparadigmen}

Zusammenfassung der Vorlesung "`Programmierparadigmen"' aus dem Wintersemester 2014.\footnote{\url{https://pp.info.uni-karlsruhe.de/lehre/WS201415/paradigmen/}}

\section{Appendix A: Haskell}

\subsection{Funktionen}

\begin{table}[h]
\begin{tabularx}{\textwidth}{l|X|X}
	\textbf{\textit{drop}} & \textit{Int -> [a] -> [a]} & Gibt die Liste ohne die ersten \textit{n} zurück\\
	\textbf{\textit{length}} & \textit{[a] -> Int} & Gibt die Länge einer Liste oder eines Texts zurück \\
	\textbf{\textit{take}} & \textit{Int -> [a] -> [a]} & Gibt die ersten \textit{n} Elemente einer Liste zurück\\
\end{tabularx}
\end{table}