\chapter{Algorithmen II}

Zusammenfassung der Vorlesung "`Algorithmen II"' aus dem Wintersemester 2014.\footnote{\url{http://geom.ivd.kit.edu/ws14_algo2.php}}


\section{Flussmaximierung}

\subsection{Flussnetzwerke}

\subsubsection{Bestandteile}
\begin{itemize}
	\item Graph mit Quelle q und Senke s: \(G(V,E)\) mit \(E \subset V^2\)
	\item Kapazitätsfunktion \(k : V^2 \rightarrow \mathbb{R}_{+0}\), wobei \(\forall e \in V^2 \setminus E: k(e) = 0\)
\end{itemize}

\subsection{Flüsse}
Ein Fluss in F ist eine Funktion \(f:V^2\rightarrow\mathbb{R}\) mit den Eigenschaften:
\begin{enumerate}
	\item \(f \leq k\)
	\item \(\forall x,y \in V : f(x,y) = -f(y,x)\)
	\item \(\forall x \in V \setminus \{ q,s \}: 0 = \sum f(x,V) := \sum_{y\in V} f(x,y)\)
\end{enumerate}

\subsubsection{Bemerkungen}
\begin{itemize}
	\item Der negative Fluss soll lediglich die Darstellung vereinfachen und ist ansonsten uninteressant.
	\item In alle Knoten au0er in \(q\) und in \(s\) fließt immer so viel positiver Fluss rein wie raus.
	\item Flussnetzwerke können zu einem Flussnetzwerk zusammengafasst werden. (mit unendlichen Zu- und Abflüssen)
\end{itemize}
