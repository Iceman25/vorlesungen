\section{Appendix A:Spickzettel}
\subsection{Flussnetzwerke}
Der maximale Fluss ist die Summe der von q ausgehenden / bei s ankommenden genutzten Kapazitäten / der minimale Schnitt durch das Netzwerk
\subsection{de Casteljau}
	$$\frac{\Delta B_0}{t} \frac{\Delta B_1}{t-1} = C(\Delta B,t) $$
\subsection{Unterteilungsalgrithmen}
	Das Differenzschema existiert nur, wenn $\alpha(z)$ den Faktor $1+z$ besitzt oder gilt $\alpha(-1) = \sum_{i \in \mathbb{Z}} {\alpha_{2i} - \sum_{i \in \mathbb{Z}}\alpha_{2i + 1} = 0$
	
	
	Symbol $\beta(z) = \frac{\alpha(z)}{1+z}$
	
	
	Existiert das zweite Differenzschema, so gilt $ \beta(-1)=0$ 	
	 
