\chapter{BWL: Finanzwirtschaft und Rechnungswesen}

Zusammenfassung der Vorlesung "`BWL: Finanzwirtschaft und Rechnungswesen"' aus dem Wintersemester 2014.\footnote{\url{https://ilias.studium.kit.edu/goto_produktiv_crs_369792.html}}



\section{Einführung}

\subsection{Investitionen}
\begin{itemize}
	\item Wertschöpfung: Projekte, deren Ertrag die Kosten der Finanzierung übersteigt
	\item Finanzierungskosten orientieren sich an der Höhe des Risikos der Projekte
	\item Investitionsobjekte: Sachinvestitionen, Immaterielle Investitionen oder Finanzinvestitionen
	\item Investitionsanlass: Gründungs-/Erstinvestitionen, Reininvestitionen oder Erweiterungsinvestitionen
\end{itemize}


\subsection{Wertschöpfung durch geschickte Finanzierung}
\begin{itemize}
	\item Ausnutzung von Fehlinformationen oder irrationalem Investorenverhalten
	\item Ausnutzung von institutionellen Verzerrungen, z.B. der unterschiedlichen Bestuerung der verschiedenen Finanzierungsformen
	\item Vermeidung von riskanten Finanzierungsstrategien
	\item Nutzung von Finanzierungsinstrumenten, die Unternehmen effizienter machen
\end{itemize}


\subsection{Finanzierungsarten}
\begin{itemize}
	\item \textbf{Außenfinanzierung}
	\begin{itemize}
		\item Fremdfinanzierung: Kredite, Anleihen
		\item Beteiligungsfinanzierung: Aktien, Anleihen
	\end{itemize}
	\item \textbf{Innenfinanzierung}
	\begin{itemize}
		\item Selbstfinanzierung: Gewinnrücklagen, Stille Reserven
		\item Sonstige Finanzierung: Rückstellungen
	\end{itemize}
\end{itemize}



\section{Bewertung von Anleihen}

\subsection{Grundlegendes}
Verbriefte, typischerweise handelbare Ansprüche gegenüber einem Schulder (Emittenten).
\begin{itemize}
	\item Dienen der langfristigen Finanzierung
	\item \textbf{Arten}
	\begin{itemize}
		\item Kuponanleihe: Periodische, fixe Zinszahlungen (Kupons) sowie endfällige Tilgungszahlung
		\item Floater: Periodisch variable Zinszahlung, die sich an kurzfristigen Zinsen (z.B. EURIBOR) orientiert mit endfälliger Tilgungszahlung
		\item Nullkuponanleihe: Keine priodischen Zinszahlungen, Tilgungsbeitrag wird am Ende ausgezahlt
		\item Verschiedene Hybride Formen
	\end{itemize}
	\item \textbf{Weitere Gestaltungsmöglichkeiten}
	\begin{itemize}
		\item Sicherheoten durch weitere Vermögensgegestände (z.B. durch Aktienbestände)
		\item Vertragliche Zusatzvereinbarungen (Covenants)
		\item Kündigungsrechte
	\end{itemize}
\end{itemize}


\subsection{Ratings}
\begin{itemize}
	\item Bonitätsbeurteilung des Emittenten oder einer einzelnen Anleihenemision durch unabhängige Agenturen
	\item \textbf{Funktionen}
	\begin{itemize}
		\item Erhöhung des Informationsstands aller Marktteilnehmer
		\item Ermöglichung des Erwerbs der Anleihe durch regulierte Institutionen
	\end{itemize}
\end{itemize}




\section{Methoden der Investitionsentscheidungen}



\section{Bewerten von Aktien}



\section{Portfoliotheorie}



\section{Grundlagen des externen Rechnungswesens}



\section{Methodik des externen Rechnungswesens}



\section{Grundlagen des internen Rechnungswesens}



\section{Kostenartenrechnung}



\section{Kostenstellenrechnun}



\section{Kostenträgerrechnung}



\section{Kennzahlen des Rechnungswesens}



\section{Appendix A: Formelsammlung}

\subsection{Anleihen}

\subsubsection{Barwert einer ewigen Rente}

\[BW = \frac{C}{r}\]

\subsubsection{Barwert einer endlichen Rente}

\[BW = \frac{C}{(1+r)^1}+..+\frac{C}{(1+r)^n} = \frac{C}{r}\cdot[1-\frac{1}{(1+r)^n}]\]

\subsubsection{Ewige Rente mit konstantem Wachstum g}

\[BW = \frac{C}{r-g}\]

\subsubsection{Endliche Rente mit konstantem Wachstum}

\[BW = \frac{C}{r-g} \cdot [1-[\frac{1+g}{1+r}]^n]\]

\subsubsection{Bewertung risikoloser Anleihen: Zerobonds}

\[y = (\frac{Nennwert}{r})^{\frac{1}{n}})-1\]
