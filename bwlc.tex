\chapter{BWL: Finanzwirtschaft und Rechnungswesen}

Zusammenfassung der Vorlesung "`BWL: Finanzwirtschaft und Rechnungswesen"' aus dem Wintersemester 2014.\footnote{\url{https://ilias.studium.kit.edu/goto_produktiv_crs_369792.html}}



\section{Einführung}

\subsection{Investitionen}
\begin{itemize}
	\item Wertschöpfung: Projekte, deren Ertrag die Kosten der Finanzierung übersteigt
	\item Finanzierungskosten orientieren sich an der Höhe des Risikos der Projekte
	\item Investitionsobjekte: Sachinvestitionen, Immaterielle Investitionen oder Finanzinvestitionen
	\item Investitionsanlass: Gründungs-/Erstinvestitionen, Reininvestitionen oder Erweiterungsinvestitionen
\end{itemize}


\subsection{Wertschöpfung durch geschickte Finanzierung}
\begin{itemize}
	\item Ausnutzung von Fehlinformationen oder irrationalem Investorenverhalten
	\item Ausnutzung von institutionellen Verzerrungen, z.B. der unterschiedlichen Bestuerung der verschiedenen Finanzierungsformen
	\item Vermeidung von riskanten Finanzierungsstrategien
	\item Nutzung von Finanzierungsinstrumenten, die Unternehmen effizienter machen
\end{itemize}


\subsection{Finanzierungsarten}
\begin{itemize}
	\item \textbf{Außenfinanzierung}
	\begin{itemize}
		\item Fremdfinanzierung: Kredite, Anleihen
		\item Beteiligungsfinanzierung: Aktien, Anleihen
	\end{itemize}
	\item \textbf{Innenfinanzierung}
	\begin{itemize}
		\item Selbstfinanzierung: Gewinnrücklagen, Stille Reserven
		\item Sonstige Finanzierung: Rückstellungen
	\end{itemize}
\end{itemize}



\section{Bewertung von Anleihen}

\subsection{Grundlegendes}
Verbriefte, typischerweise handelbare Ansprüche gegenüber einem Schulder (Emittenten).
\begin{itemize}
	\item Dienen der langfristigen Finanzierung
	\item \textbf{Arten}
	\begin{itemize}
		\item Kuponanleihe: Periodische, fixe Zinszahlungen (Kupons) sowie endfällige Tilgungszahlung
		\item Floater: Periodisch variable Zinszahlung, die sich an kurzfristigen Zinsen (z.B. EURIBOR) orientiert mit endfälliger Tilgungszahlung
		\item Nullkuponanleihe: Keine priodischen Zinszahlungen, Tilgungsbeitrag wird am Ende ausgezahlt
		\item Verschiedene Hybride Formen
	\end{itemize}
	\item \textbf{Weitere Gestaltungsmöglichkeiten}
	\begin{itemize}
		\item Sicherheoten durch weitere Vermögensgegestände (z.B. durch Aktienbestände)
		\item Vertragliche Zusatzvereinbarungen (Covenants)
		\item Kündigungsrechte
	\end{itemize}
\end{itemize}


\subsection{Ratings}
\begin{itemize}
	\item Bonitätsbeurteilung des Emittenten oder einer einzelnen Anleihenemision durch unabhängige Agenturen
	\item \textbf{Funktionen}
	\begin{itemize}
		\item Erhöhung des Informationsstands aller Marktteilnehmer
		\item Ermöglichung des Erwerbs der Anleihe durch regulierte Institutionen
	\end{itemize}
\end{itemize}



\section{Methoden der Investitionsentscheidungen}

\subsection{Capital Budgeting}
Der Entscheidungsprozess über die Durchführung von Investitionen.

\subsubsection{Kapitalwert (Net Present Value) einer Investition}

\[Kapitalwert = -Anfangsauszahlung + \frac{Barwert}{1+r}\]

\begin{itemize}
	\item Entscheidungsregel: Führe die Investition durch, wenn Kapitalwert positiv.
	\item Ist das einzusetzende Kapital knapp und infolgedessen können nicht alle Projekte mit positivem Kapitalwert durchgeführt werden, liefert die Kapitalwertmethode keine Lösung des Auswahlproblems.
\end{itemize}

\subsubsection{Methode der Kapitalwertrate}
Lediglich für unabhängige Projekte geeignet.

\[KWR = \frac{Barwert}{Anfangsauszahlung}\]

\subsubsection{Amortisierungsrechnung}
\begin{itemize}
	\item Ziel: Berechnung der Zeitspanne, in der die Anfangsauszahlung wieder in Form kumulierter Zahlungen zurückgeflossen ist.
	\item Entscheidungsregel: Bestimmte, maximale Amortisierungsdauer
	\item \textbf{Vorteile}
	\begin{itemize}
		\item Schnell und einfach anzuwenden
		\item Fehlentscheidungen werden schnell offenbar, z.B. wenn geplante Zahlungsströme nicht eintreten
		\item Unternehmen können zeigen, dass sich Investitionen lohnen
	\end{itemize}
	\item \textbf{Nachteile}
	\begin{itemize}
		\item Keine Berücksichtigung des Zeitpunkts der Zahlungen (später vs. heute)
		\item Keine Berücksichtigung von späteren Zahlungen
		\item Willkürliche Bestimmung der gewünschten Amortisationsdauer
	\end{itemize}
\end{itemize}

\subsubsection{Interne Zinssatzmethode}
\begin{itemize}
	\item Der interne Zinssatz ist derjenige Zinssatz, bei dem der Kapitalwert einer Investition 0 beträgt.
	\item Entscheidungsregel: Investiere, wenn der interne Zinssatz größer als der geforderte Zinssatz ist.
	\item Berechnung: Setze Kapitalwert gleich Null
	\item Vorteile: Simple Kennzahl als Diskussionsgrundlage; Berücksichtigung von mehrperiodigen Zahlungen
	\item Nachteil: Schwierige Berechnung bei sehr langen Projekten
\end{itemize}



\section{Bewerten von Aktien}

\subsection{Empirische IPO-Phänomene}
\begin{itemize}
	\item Underpricing: Bookbuilderpreis meist weit unter erstem Handelskurs
	\item Zyklizität: Anzahl der IPOs abhängig vom aktuellen Börsenklima
	\item Kosten: Transaktionskosten relativ hoch
	\item Langfristige Performance: Erst 3-5 Jahre meist eher durchschnittlich
\end{itemize}


\subsection{Kapitalbeschaffung durch Kapitalerhöhung}
\begin{itemize}
	\item Ordentliche Kapitalerhöhung: Ausgabe von jungen Aktien zur Beschaffung von neuem Eigenkapital
	\item Bedingte Kapitalerhöhung: Kapitalerhöhung durch Gebrauch von Umtausch- oder Bezugsrechten
	\item Genehmigtes Kapital: Ermächtigung durch die Hauptversammlung, für befristete Zeit eigenmächtig das Kapital erhöhen zu können
\end{itemize}


\subsection{Dividenden}
\begin{itemize}
	\item Im perfekten Kapitalmarkt sinkt der Aktienkurs um den Betrag der ausgezahlten Dividende
	\item Dividend Smoothing: Langfristig stabile Dividenden, da Unternehmen sisch schwer tun, diese zu verändern
	\item \textbf{Signalwirkung}
	\begin{itemize}
		\item Signal des Managements bezüglich der Gewinnerwartung (Dividendenerhöhung/-senkung)
		\item Es gibt allerdings auch Ausnahmen
	\end{itemize}
	\item Dividenden sind meist mit einem höheren Steuersatz belegt als Kapitalgewinne, die Investoren durch den Verkauf gestiegener Aktien verdienen können $\rightarrow$ anpassbar an die Steuerpreferenz der Aktionäre
\end{itemize}


\subsection{Bewertung von Aktien}
\begin{itemize}
	\item Aktiengenerierter Zahlungsstrom unsicher $\rightarrow$ Bewertung der Aktie durch Diskontierung des erwarteten Zahlungsstrom
	\item \textbf{Quellen für Zahlungen}
	\begin{itemize}
		\item Dividenden
		\item Aktienverkäufe
	\end{itemize}
\end{itemize}



\section{Portfoliotheorie}
\begin{itemize}
	\item Fundament der modernen Kapitalmarkttheorie: Abwägung zwischen Ertrag und Risiko
	\item Idee: Risikominimierung durch Diversifikation
\end{itemize}


\subsection{Portefeuilles}
Investoren betrachten keine isolierten Wertpapiere sondern Portefeuilles.

\subsubsection{Erreichbare Kombinationen effektiver Portefeuilles}
\begin{itemize}
	\item Berechnung des Globalen Varianzminimalen Portefeilles (GMVP)
	\item Die Porteifeilles oberhalb des GVMP sind effizient
	\item Konkrete Wahl abhängig von der Risikoeinstellung
\end{itemize}

\subsubsection{Erreichbare Kombinationen mit Tangentialportefeuille}
\begin{itemize}
	\item Zusätzliches, risikoles Papier; darstellbar als Gerade
	\item Berechnung des Schnittpunkts der Kurven
	\item Durch Verschuldung mit dem risikolosen Papier sind höhere Investitonen möglich
\end{itemize}


\subsection{CAPM}
\begin{itemize}
	\item Jeder Investor hält Kombination aus Tangentialportefeuille und risikolosem Instrument (Aufteilung investorspezifisch)
	\item Anpassung der Kurse, so dass Markträumung stattfindet
\end{itemize}

\subsubsection{Zerlegung des Gesamtrisikos}
\begin{enumerate}
	\item Risikobeitrag zum Gesamtrisiko: Höhere Rendite \(cov(r_j,r_M)\)
	\item Unternehmensbezogenes Risiko: Keine höhere Rendite \(\sigma^2_{\epsilon_j}\)
\end{enumerate}



\section{Grundlagen des externen Rechnungswesens}



\section{Methodik des externen Rechnungswesens}



\section{Grundlagen des internen Rechnungswesens}



\section{Kostenartenrechnung}



\section{Kostenstellenrechnun}



\section{Kostenträgerrechnung}



\section{Kennzahlen des Rechnungswesens}



\section{Appendix A: Formelsammlung}

\subsection{Anleihen}

\subsubsection{Barwert einer ewigen Rente}
\[BW = \frac{C}{r}\]

\subsubsection{Barwert einer endlichen Rente}
\[BW = \frac{C}{(1+r)^1}+..+\frac{C}{(1+r)^n} = \frac{C}{r}\cdot[1-\frac{1}{(1+r)^n}]\]

\subsubsection{Ewige Rente mit konstantem Wachstum g}
\[BW = \frac{C}{r-g}\]

\subsubsection{Endliche Rente mit konstantem Wachstum}
\[BW = \frac{C}{r-g} \cdot [1-[\frac{1+g}{1+r}]^n]\]

\subsubsection{Bewertung risikoloser Anleihen: Zerobonds}
\[y = (\frac{Nennwert}{r})^{\frac{1}{n}})-1\]


\subsection{Anleihen}

\subsubsection{Investor mit Einjahreshorizont}
\[r = \frac{Div_1+P_1}{P_2}-1\]

\subsubsection{Divided Discount Model}
\[P_0 = \sum_{t=1}^{N} \frac{Div_t}{(1+r_E)^t}\]

\subsubsection{Divided Discount Model mit unterschiedlichen Wachstumsraten}
Erweiterung um ewige Rente.
\[P_0 = \sum_{t=1}^{N} \frac{Div_t}{(1+r_E)^t} + \frac{1}{(1+r_E)^N}\cdot\frac{Div_{N+1}}{r_E-g}\]


\subsection{Portfoliotheorie}

\subsubsection{Aktienrendite}
Bei verschiedenen Papieren mit linearer Gewichtung.

\[r = \frac{Div + K_1 - K_0}{K_0}\]

\subsubsection{Standardabweichungen der Rendite}
Bei verschiedenen Papieren auch hier mit linearer Gewichtung unter der Wurzel.

\[Std = \sqrt{Var} = \sqrt{\frac{(R_1-R)^2+..+(R_N-R)^2}{N-1}}\]

\subsubsection{Erwartungswert der Portefeuillerendite}
\[\mu_w = w_1\mu_1 + w_2\mu_2\]

\subsubsection{Risiko der Portefeuillerendite}
\[\sigma^2_w = w^2_1\sigma^2_1+w^2_2\sigma^2_2+2w_1w_2cov(r_1,r_2)\]

\[cov(r_1,r_2) = \sigma_1\sigma_2\rho_{12}\]

\subsubsection{CAPM: Kapitalmarktlinie}
\[\mu_W = r + \frac{\mu_M-r}{\sigma_M}\cdot\sigma_W\]

\subsubsection{CAPM: Einzelne Wertpapiere im Gleichgewicht}
\[\mu_j = r + \frac{\mu_M-r}{cov(r_M,r_M)}\cdot cov(r_j,r_M) = r + (\mu_M-r)\cdot \beta_j\]
\[\beta_j = \frac{cov(r_j,r_M)}{cov(r_M,r_M)}\]
