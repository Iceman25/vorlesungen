\section{Allgemeines}
	\subsection{Welche Themen wurden in der Vorlesung behandelt?}
		\begin{itemize}
			\item Klassifikation von Geräten
			\item Privatsphäre im IoE
			\item Betrachtung des IoE in Bezug auf das OSI-Modell
			\item Sicherheitsaspekte
		\end{itemize}
	
	\subsection{Auf welche Probleme stößt man im IoE? Warum müssen diese extra behandelt werden?}
		\begin{itemize}
			\item Möglichste Energiesparend, da Batteriebetrieben
			\item Sehr beschränkte Rechenleistung
			\item Kommunikationswege nicht zuverlässig, Semi-Broadcast
			\item Meist keine Infrastruktur vorhanden, wenig zentrale Infrastruktur
			\item Geringe Bauform und muss günstig sein
			\item Greif tief in die Privatsphäre des Menschen ein, Sammelt private Daten.
			\item Omnipräsenz?
		\end{itemize}
		
	\subsection{Welche Unterschiede und Besonderheiten gibt es?}
		\begin{itemize}
			\item Dezentral
			\item Selbstorganisieren
			\item Limitierte Ressourcen
			\item Unzuverlässiger Ressourcenkanal
			\item Unsicher, knoten können zerstört oder ausgelesen werden.
		\end{itemize}
		
	\subsubsection{Was muss man deswegen besonders beachten?}
		Durch die Allgegenwärtigkeit der Knoten muss besonders auf die Privatsphäre geachtet werden.
		
	\subsubsection{Privatsphäre was ist das?}
	Die Möglichkeit vertraulich zu kommunizieren und selbst darüber zu bestimmen, wer welche Daten über mich erhält.
	
	\subsubsection{Warum ist Privatsphäre wichtig? Wo ist das Problem beim Datenschutz?}
	sensitive Daten?
	
	\subsubsection{Was sind die Prinzipien des Datenschutzes?}
	Datensicherheit, Datensparsamkeit, Rechtmässigkeit, Transparenz, Nutzerrechte, Kontrolle.
		
	\subsubsection{Was gibt es für Schutzziele?}
	
	\subsubsection{Wie realisieren wir Sicherheit im IoE?}
	
	\subsubsection{Wir hatten eine Wetterstation als Beispiel, wie funktioniert das?}
	Wetterstation sammelt Daten und überträgt diese an die Server des Herstellers. Kunde greift mit seinem Smartphone auf die gesammelten Daten auf den Servern des Herstellers zu. Die Daten liegen nicht beim Kunden sondern beim Hersteller in fremder Hand.
	\paragraph{Wo liegt hier das Problem?}
		Es wird unter anderem die Lautsärke gemessen, die lässt eventuell sogar Sprachaufzeichnungen zu. Desweiteren können über diese Daten auf das Verhalten der Bewohner schließen lassen.
	\subsubsection{Wo liegt das Problem bei Smart Metering?}
	Aus dem hochfrequenten sampling der Verbrauchswerte lassen sich ebenso Verhaltensmuster schließen.
	\paragraph{Warum will man das dann?}
		Durch die hochfrequenten Messwerte lassen sich Geräte schalten wenn Stromüberschüsse vorhanden sind außerdem kann der Netzbetreiber mit diesen Informationen sein Stromnetz besser regeln und weiß wann er mehr oder weniger erzeugen muss.
	\paragraph{Womit lässt sich dieses Problem umgehen?}
		SMART-ER
	\subsection{Was ist das Problem bei vielen Geräten auf kleinem Raum, und was macht man dagegen?}
	Das Problem sind Kollisionen und der damit verbundene Energieverbrauch durch Erkennung und Retransmission. Das Routing wird komplexer. Die Gegenmaßnahme ist Topologiekontrolle.
	
	\subsection{Was hat bei Sensorknoten den höchsten Energieverbrauch?}
	Funkschnittstelle und, wenn vorhanden, Display.
	
	\subsection{Welche Netztopologien kommen im IoE vor?}
	\begin{itemize}
		\item Einzellnes Gateway
		\item Mehrere Gateways (Redundanz)
		\item Ad\- Hoc\- Netz
	\end{itemize}
		
\section{Routing}
	\subsection{Welche Verfahren gibt es?}
		Probalistische, lokalisationsbasierte und inhaltsbasierte Verfahren.
		
	\subsection{Welches inhaltsbasierte Verfahren hatten wir?}
		Direct Diffusion
	
	\subsubsection{Wie funktioniert Direct Diffusion?}		
		Sender broadcastet Interesse nach bestimmten Daten in Form von Attribut-Wert Paaren. System speichert Richtung aus der ein Intresse kam in Form von Gradienten. Sensoren mit entsprechenden Daten schicken die Daten an den Sender. Sobal der Sender merkt, dass die Daten verfügbar sind, startet er eine Reinforcment Phase zur Verstärkung der Gradienten.
		\subsubsection{Genaures im Detail } %TODO
		

\section{MAC}
	\subsection{Welche Verfahren hatten wir auf der MAC-Schicht?}
		S-MAC, B-MAC und 802.15.4 
		
		\subsubsection{Wie lassen sich diese Klassifizieren?}
		Zentral(802.15.4) Dezentral(S-Mac), Synchron(S-Mac) Asynchron(B-Mac) %TODO
		
	\subsection{Wie funktioniert S-MAC?}
	%TODO	
	
	\subsubsection{Welche Erweiterungen gab es bei S-MAC und wie funktionieren diese?}
		\begin{itemize}
			\item ALP:
			\item MP (Message Passing):
		\end{itemize}
		
	\subsection{Wie funktioniert B-MAC?}
	
	\subsection{Wie funktioniert 802.15.4 im Beacon Modus?}
	Duty Cycling, gibt das Verhältnis zwischen wach und schlafphasen an.
	\subsubsection{Wie erhält man im Beacon Modus einen garantierten Zeitschlitz?}
	%TODO Zeit Paket Diagram aus der Vorlesung zeichen
	
	


\section{Sicherheit}
	\subsection{Warum wollen wir vertraulich kommunizieren}
	
	\subsection{Reicht Verschlüsselung um die Privatsphäre zu schützen}
	Nein, Beispiel SMART\- METER
	\subsection{Was haben wir besprochen?}
	Single Mission Key, Zufallsverteilte Schlüssellisten und Key Infection
	
	\subsection{Wie funktioniert Single Mission Key?}
	Ein Schlüssel für alle Systeme, alle nutzen diesen für Verschlüsselung und Integritätssicherung. Problem ist, dass ein kompromitiertes Gerät reicht umd die gesamte Sicherheit zu brechen.
	\subsubsection{passiert wenn ein neues System hinzukommt?}
	Es wird mit dem selben Single Mission Key konfiguriert und kann dann mit dem Rest des Netzes kommunizieren.
	
	\subsection{Was ist EGLI?}
	Eschenauer Gligor
	\subsubsection{Wei funktioniert EGLI?}
	Es gibt eine globale Schlüsselliste und lokale zufällige Teilmengen als Keyrings.
	
	%TODO Digagramm des Schlüsselaustauschs
	A sendet Zufallszahl, B antwortet mit verschlüsselter Zahl mit allen Schlüsseln, damit kennt A nun die Schlüssel(durch Entschlüsseln mit eigenen Schlüsseln) 
	
	Keypool P wird vom Benutzer zufällig erzeugt. Jedes System erhält einen Keyring R mit einer Teilmenge von P. Es wird gezeigt, dass mit guter Wahl von P und R eine Wahrscheinlichkeit von mehr als 99\% erreicht werden kann, dass zwei Systeme einen gemeinsamen Schlüssel in ihren Keyring haben. 
	\paragraph{Ist EGLI zuverlässig?}
	Nein, da es keine E2E Quittungen gibt.
	\subsubsection{Wie Kann man den Aufwand minimieren?}
	Wert selbst mitschicken und damit nur mit richtigem entschlüsseltem Antworten.
	
	\subsubsection{Was ist Zuverlässigkeit?}
	Vollständige Übertragung in der korrekten Reihenfolge ohne Dupplikate.
	\paragraph{Wie stellen wir Zuverlässigkeit sicher?}
		Time, Quittungen, Sequenznummern, FEC, E2E Quittungen, Hop by Hop Quittungen.
	\subsubsection{Was ist der Unterschied zwischen E2E und Hop 2 Hop Kommunikation?}
	
	\subsection{Was ist Privatsphäre?}
	Right to be let alone, hat sich zu Recht auf Bestimmung über Verwendung persönlicher Daten entwickelt.
	\subsubsection{Beispiele für Geräte die persönliche Daten aufzeichnen?}
	Smart Watch, SmartPhone, Thermometer, Smart Meter...
	\subsubsection{Was ist bei Stromzählern problematisch? Die werden doch heutzutage auch schon abgelesen?}
	Die zeitliche Auflösung ist das Problem, bisher gab es einen Verbrauchswert pro Jahr, jeder sind Werte im Bereich von weniger als einer Sekunde möglich. Dies lässt Rückschlüsse auf das private Leben schließen.
	\subsubsection{Warum benötigt man Smart\- Meter?}
	Dezentrale Steuerung des Stromnetzes v.a. bei volatilen regenerativen Energiequellen (dezentral Energiequellen oder Verbraucher passend ein\-  und ausschalten, z.B. schaltet sich die Waschmaschine erst bei lokalem Stromüberschuss an.)
	\subsubsection{Wie kann man Schutz der Privatsphäre bei Smart\- Metern absichern?} 
	\begin{itemize}
		\item Grundmaßnamen sind immer Pseudoanonymisierung und Verschleierung (zeitliche und örtliche Auflösung)
		\item Pseudoanonymisierung ist jedoch problematisch da hier eine Trusted Third Party notwendig ist
		\item Zeitliche Verschleierung verhindert jedoch die effiziente Nutzung der Daten
		\item Besser ist hier eine örtliche Verschleierung durch Gruppenbildung (SMART\- ER)
	\paragraph{Wie funktioniert SMART\- ER?}
	%TODO Diagramm einfügen
	Gruppenbildung und Zufallszahlen austauschen
	
	Summe und Menge der Kommunikationspartner zurückgeben( weil Gültigkeit der Werte davon abhängt, ob Kommunikationspartner gültige Werte geliefert haben)
	
	Das Problem ist die Gruppeneinteilung durch Anbieter oder Trusted Third Party, so könnte ein Anbieter einfach jeden Zähler in eine eigene Gruppe zuweisen und das ganze Verfahren wäre sinnlos. Alternativen sind Smart Meter Speed\- dating(aber: Woher bekommen die Knoten vertrauenswürdig die gloable Liste aller Teilnehmer?) oder dezentrales Aggegieren in Overlay\- Netzen.
	\end{itemize}
	
	\subsection{Was ist das besondere im Bezug auf traditionelle Sicherheit?}
	\begin{itemize}
		\item Hauptproblem: Ressourcenknappheit, d.h. asymmetrische Verfahren sind problematisch, Schlüsselaustausch muss mit symmetrischen Verfahren implementiert werden.
		\item Oft gibt es keine zentrale Infrastruktur
		\item Schlüsselaustauschverfahren für klassische Sensornetzen: EGLI, Key Infection: DICE als schlankeres DTLS
	\end{itemize}
	
	\subsection{Wie funktioniert Key Infection?}
	%TODO Diagramm des Schlüsselaustausches zeichnen
	A sendet $k_i$, B sendet $Enc_{k_i}(k)$ zurück, dies ist nun der Sitzungsschlüssel.
	Nur wer in direkter Reichweite von A war, konnte $k_i$ abhören und kennt damit den Sitzungsschlüssel, damit muss der Angreifer sehr viele räumlich verteilte Systeme korrumpieren.
\section{noch nicht zugeordnet}
	\subsection{Wie kriegt man ein Sensornetz ans Internet angebunden?}
		Mit 6LoWPAN, angepasste Version von IPv6 für das IoT. Bildet Adaptionsschicht zwischen 802.15.4 und IPv6.
		\subsubsection{Was ist 6LoWPAN?}
		IPv6 für schwache Knoten, z.B. Sensornetzen. Ermöglicht Anbindung an das "normale" Internet. Bietet bspw. Header Compression. Ein Gateway setzt dann von 6LoWPAN in normales IPv6 um.
		\subsubsection{Wie funktioniert 6LoWPAN?}
		komprimierung/dekomprimierung am Edge-Router
		\subsubsection{Welche Arten von Adressierung gibt es in IPv6?}
		Unicast, Multicast und Anycast
		\subsubsection{Wie sieht eine 6LoWPAN Datenpaket aus?}
		Dispatch\- Byte (Header\- Typ/Fragmentierung), danach Paketheader
		Paketheader kann z.B. komprimierter IP\-  bzw. UDP\- Header sein, in diesem Fall enthält er ein Bitfeld, welches angibt welche Teile des Headers komprimiert werden.
		\subsubsection{Was ist das Dispatch/Compress Byte?}
		Das Dispatch Byte ist ein verpflichtendes "Headerfeld", das angibt ob es sich um einen komprimierten/unkomprimierten IPv6 Header handelt oder ob das Paket fragmentiert ist. Danach kommt nochmal ein Byte das anzeigt welches Feld des IP Heads komprimiert ist.
		\subsubsection{Welche Header Felder werden komprimiert und wie?}
		Direkt aus PAN bzw. MAC\- Adresse abgeleitet, konkateniert mit Netzpräfix, d.h. Präfix kann bei lokalen Adressen weggelassen werden, bei direkter Kommunikation kann die Adresse sogar ganz weg gelassen werden, das sie bereits im MAC\- Header steht.
		Alle Felder die bereits duch MAC Felder abgedeckt werden und ansonsten redundante Informationen darstellen würden.
		Verkehrsklasse/Flow Label, Version, Addressen, Länge...
		Versionsfeld wird komprimiert, da immer 6.
		
		Traffic-Control und Flow Label sind meist 0
		
		Adresse bei Link-Lokaler Kommunikation,  volle Adressen nur, wenn vorher ein Kontext etabliert wurde)
		
		Next Header
		
		Länge aus Schicht 2 
		
		\paragraph{Wie sieht ein komprimierter Header aus?}
		Mehrere Bits die angeben, ob Adresse, Länger, NextHeader etc... komprimiert oder unkomprimiert vorliegen, gefolgt von den entsprechenden Feldern.
		
		\subsubsection{In welche Richtung setzt das Gateway die komprimierten Header um?}
		Von IPv6 zu 6LoWPAN: komprimierung
		Von 6LowPan zu IPv6: dekomprimierung
		
		\subsubsection{Gibt es noch andere Header die komprimiert werden können?}
		Ja, UDP. Hierbei werden weniger Ports benutzt und die Länge auf Schicht 3 berechnet. Die Sequenznummer kann komprimiert werden, das UDP nicht zuverlässig ist.
		\paragraph{Wofür braucht man überhaupt Ports?} Bindung von Programmen an die Netzwerkadresse.
		\paragraph{Wieso benötigt UDP hier weniger Ports?} UDP ist kein zuverlässiges Protokoll und belegt deswegen keinen Port um auf eine Bestätigung des Empfangs zu hören.	Ausserdem kann die Anzahl der verfügbaren Ports reduziert werden da so ein ressourcenarmes Gerät vermutlich keine $2^{16}$ Anwendungen ausführen wird.	
		
		
		
		
		
		\subsubsection{Wie lang ist eine IPv6\- Adresse?}
		128 Bit. Die ersten 64Bit sind der Präfix, welche bei link-lokaler Kommunikation nicht benötigt wird und die zweiten 64 Bit sind die Interface ID, z.B. kann hier die MAC-Adresse des Adapters verwendet werden.
		\subsubsection{Wie lange ist ein IPv6\- Header?}
		40 Byte
		\subsubsection{Wie groß ist ein 6LoWPAN Paket?}
		\subsubsection{Wie groß ist ein 801.15.4 Frame?}
		
		\subsubsection{Wie funktioniert das Routing bei 6LoWPAN?}
		RPL???? DAG? NodeRank? Wurzel? RootRank??
		\subsubsection{Wie funktioniert Address Autoconfig?}
		\subsubsection{Wie funktioniert Fragmentierung?}
	
	\subsection{Welche Kommunikationsformen gibt es in WPANs?}
		Unicast, Multicast und Concast
		\subsubsection{Was ist Unicast?}
			\paragraph{Was ist HHR?}
			\paragraph{Was ist E2ER?}
		\subsubsection{Was ist Multicast?}
		\paragraph{Gibt es bei Multicast ein spezielles Verfahren bezüglich der Zuverlässigkeit, wenn ja welches und wie funktioniert es?}
			Pump Slowly Fetch Quickly: %TODO
			\subparagraph{Wie funktioniert das?}
			Szenario: Sender sendet, System leitet nach Verzögerung weiter
			\subparagraph{Leiten sie immer weiter?}
			Nein, wenn mindestens 3 andere bereits weitergeleitet haben, dann nicht
			\subparagraph{Und ansonten immer?}
			Neun, nur wenn das System alle Sequenznummern erhalten hat. Fehlet eine Sequenznummer, so wird diese erstmal mit einem NACK an alle Nachbarn angefragt. Nachbarn warten zufällige Zeit pro Sequenznummer und verschicken das Paket nur nochmal, falls nicht schon ein anderes System auf das NACK reagiert hat.
			\subparagraph{Wann und wie werden NACKs verbreitet?} 1Hop Umgebung (semi-bc)
			\subparagraph{Wann werden Daten gepumpt?} Probalistisches Multicast bis zu gewissen grenzen, kein Fluten
			\subparagraph{Wenn ein Knoten anhand der Sequenznummer feststellt, dass ihm ein Paket fehlt, muss er dann seine Nachbarn kennen um das NACK zu senden?}
			Nein, das NACK wird per Broadcast mit TTL=1 gesendet.
			\subparagraph{Haben wir damit jetzt 100\% zuverlässigkeit?}
			Nein, kommt keine Dateneinheit mit falscher Sequenznummer, so sieht auch kein Knoten die Notwendigkeit ein NACK zu senden.
			\subparagraph{Wie addressiert der Sender die Empfänger?}
			Wenn alle erreicht werden sollen, wird per Broadcast transmittet, soll nur eine Teilgruppe der Teilnehmer erreicht werden, so wird per Multicast addressiert %TODO ähm?
			\subparagraph{Kann ich bei einem Broadcast die Adresse einfach weglassen?}
			Nein man benutzt eine Broadcastadresse.
			
		\paragraph{Was ist ESRT und wie funktioniert es?}
		Senke broadcastet gewünschte Rate über starke Sendeleistung, da es meistens am Stromnetz hängt.
			
		
		\subsubsection{Was ist Concast?}
		Viele Sender senden zu einem Empfänger(Senke)
		\paragraph{Welches Protokoll haben wir hier erwähnt?}
			ESRT 
			\subparagraph{Was ist ESRT?}
			
			\subparagraph{Wie funktioniert ESRT?}
			
			\subparagraph{Wie wird die neue Senderate publiziert?}
			Broadcast von der Senke an alle Quellen. Dafür muss sie aber auch alle mit ihrer Senderstärke erreichen können.
			
		\subsubsection{Was ist ein Semi\- Broadcast Medium?}
		Funk, nur eine Teilmenge aller Empfänger liegt in der Sendereichweite
		
	\subsection{Wie funktioniert LEACH?}
	Zeit wird in Runden diskreditiert, in jeder Runde werden zufällig $N \times P$ Cluster\- Heads bestimmt.
	Es gibt jeweils $\frac{1}{P}$ Runden zusammengefasst, innerhalb dieser Runden wächst die Wahrscheinlichkeit, dass sich ein Knoden als Cluster\- Head definiert, bis auf 1 an. 
	
	Nach der Wahl der Cluster\- Heads: Andere Knoten wählen ein Cluster
	
	%TODO ich hab keine Ahnung
	
	\subsubsection{Wie wird in LEACH Energie gespart?}
	Nur die Cluster\- Heads müssen jederzeit mit Gateway o.ä. kommunizieren (hohe Sendeleistung).
	Die Cluster\- Heads stellen einen Zeitplan auf und die anderen Knoten kommunizieren immer nur mit den Cluster\- Heads (Sterntopologie) und müssen nur wach sein, wenn es der Zeitplan vorsieht.
	
	\subsection{Was ist CoAP}
	CoAP ist eine leichtgewichtige Variante von HTTP(kompaktes binäres Protokoll, dass auf UDP aufbaut).
	Anfragetypen, zwei Schichten, Pakettypen.
	
	\subsection{Mac-Protokolle? Was anstatt WLAN?}
	IEEE 802.15.4, B-MAC, S-MAC
	
\section{802.15.4}
	\subsection{Was ist 802.15.4}
	PHY + MAC Protokoll im IoE
	Soll WLAN/Bluetooth ersetzen.
	
\section{Geräteklassen}
	\subsection{Was sind Nanonetze?}
	Kleinste vernetzte Nanomaschinen die jeweils nur eine Aufgabe wie Speichern, Messen, Berechnen oder Manipulieren ausführen können. Nanonetze benutzen Molekulare Kommunikation anstatt elektromagnetische.
	
	\subsection{Was ist Smart Dust?}
	Sehr kleine beschänkte Hardware im Nano bis Millimeter Bereich. Die in kooperation zusammen wirken. Bisher nur Forschungsthematik. 
	
	\subsection{Was sind klassische Sensornetze?}
	Sehr viele kleine auf Einzelanwendungen maßgeschneiderte Systeme. Sie sind batteriebetrieben und haben nur geringe Hardwareressourcen.
	
	\subsection{Was ist Physical \& Embedded Computing}
	Flexible Systeme aus Standardhardware, welche kostengünstig ist und eine hohe Energieeffizient aufweist. Im gegensatz zu klassischen Sensornetzen können diese Geräte auch einen ständigen Stromanschluss haben.
	
	\subsection{Was ist Smart\- und Submetering}
	Die zeitnahe Erfassung (und Steuerung) von Energieverbäuchen (~15min). Soll die effizientere Nutzung von Ressourcen ermöglichen. Das deutsche Modell (BSI) sieht viele Zähler und einzellne Gateways vor. 
	
	\subsection{Was ist Smart Home?}
	Hausautomation und monitoring durch Sensornetze. 
	Z.B. zur Rolladen oder Licht Steuerung.
	Dabei kommen Probleme wie einfachheit vs. Sicherheit auf. 
	\subsubsection{Wie sieht es mit Smart Home und Privacy aus?}
	Die Sensoren sind tief in den Lebensraum verflochten und können höchst private Daten sammeln.
	\subsection{Wie sieht es mit Robotern und mobilen Plattformen aus?}
	Hier handelt es sich um eine mobile Plattform mit Sensoren udn Aktoren zur Messung vor Ort.
	Sie könenn in lebensfeindlichen oder unzugänglichen Umgebungen eingesetzt werden.
	\subsection{Was sind Wearables?}
	Am Körper Tragbare Sensoren die das persönliche Verhalten, Aktivitätsdaten und Vitalwerte erfassen.
	Die Auswertung und Anzeige ist auf diesen Geräten nur rudimentär möglich.
	\subsection{Was sind Smart Phones?}
	Sie stellen die heutige Schnittstelle zwischen Mensch und IoE da, sie erlauben die Steuerung von Remote-Geräten.
	Sie bestehen aus leistungsfähiger Hardware und setzen die Existenz von leistungsfähiger Kommunikationsinfrastruktur voraus.
	\subsection{Was sind Single\- Board Computer?}
	Kostengünstige Hardware für die Entwicklungs und Prototypen Umgebung.
	Sie haben eine große Anzahl an verschiedener Bus Systeme und Anschlussmöglichkeiten.
	
	\subsection{Was ist Industrie 4.0 \\ Industrial Internet?}
	Eine heterogene Umgebung an unterschiedlichsten Sensoren. Beinhaltet Leistungsfähige Steuer\- und Regelsysteme.
	Drahtlose Sensorik über Batterie oder Energy Harvesting.
	Anforderungen: Zuverlässigkeit, Robustheit, Langlebigkeit.
	
\section{Privatsphäre}
	\subsection{Was ist RFID und wo liegen die Herausforderungen für den Datenschutz?}
	RFID steht für Radio Frequency Identification. Drahtloste Übertragung von Informationen zwischen einem Transponder und einer Basistation.
	Das Problem ist das über diese Transponder eine eindutige Identifikation eines Objekts/Produkts oder einer Person möglich ist. Dieser Transponder kann auch aus großer Entfernung ausgelesen werden.
	\paragraph{Gibt es Lösungsansätze?}
	Ein möglicher Lösungsansatz wäre einen RFID Tag in einem Produkt an der Kasse duch einen "Kill" Befehl zu zerstören.
	Desweiteren könnten Kryptografische Mechanismen zur Authentifizierung genutzt werden.
	Physikalisch könnte auch einfach die Antenne abgerissen werden.
	
	\subsection{Wie sieht es mit dem Schutz der Privatsphäre aus?}
	Schutz des Persönlichkeitsrechts bei der Verarbeitung personenbezogener Daten 
	\subsubsection{Welche Daten sind schützenswert?}
	Laut BDSG sind alle Personenbezogenen Daten schützenswert allerdings sind im Kontext des Internet of Everything auch alle Metadaten schützenswert. Das heißt wer mit wem wann kommuniziert etc..
	
	\subsection{Was sind die Säulen zum Schutz der Privatsphäre?}
	\begin{itemize}
		\item Regulierung: Bundesdatenschutzgesetz etc..
		\item Selbstreuglierung: z.B. Gütesigel die den vertraulichen Umgang mit Daten bescheinigen.
		\item Selbstschutz: Privat Enhancing Technologies z.B. TOR
	\end{itemize}
	
	\subsection{Was sind allgemeine Schutzziele?}
	Anforderungen die Erfüllt werden müssen um schützenswürdige Güter vor Bedrohung zu schützen
	\begin{itemize}
		\item Confidentiality(Vertraulichkeit): Ermöglicht keine unautorisierte Informationsgewinnung
		\item Integrity(Integrität)
		\begin{itemize}
			\item Starke Integrität: Es ist unmöglich Daten unautorisiert zu verändern.
			\item Schwache Integrität: Es ist unmöglich Daten zu verändern ohne das es bemerkt wird.			
		\end{itemize}
		\item Avaliability(Verfügbarkeit): System beleibt Verfügbar und gewährt keine Einschränkung durch unautorisierten Zugriff.
	\end{itemize}

	\subsection{Spezielle Schutzziele für die Privatsphäre?}
	\begin{itemize}
		\item Unverkettbarkeit: Daten aus unterschiedlichen Kontexten sind nicht miteinander in Bezug zu setzen, z.B. durch Datenvermeidung oder Anonymisierung.
		\item Transparenz: Die Verarbeitung von Daten ist nachvollziehbar und überprüfbar. 
		\item Intervenierbarkeit: betroffene Personen können über die Erfassung und Verarbeitung ihrer Daten selbst bestimmen.
		
\end{itemize}		
	
	\subsection{Was ist ein Vertrauensmodell?}
	Vertrauen: Bewertung wie sich eine Entität in einem konkreten Sacherverhalt verhalten wird, hier die Annahme, dass Entität kein Angreifer ist.
	\begin{itemize}
		\item Vollständiges Vertrauen: uneingeschränktes Vertrauen aller Entitäten des Systems
		\item Keinerlei Vertrauen: Alle Entitäten potentielle Angreifer, PET notwendig
		\item Vertrauen in zentrale Instanz:Trusted third party
		\item Verteiltes Vertrauen: Nutzer vertraut lediglich, dass eine Teilmenge der beteiligten Entitäten nicht kooperiert.
		
	\end{itemize}
	\subsubsection{Was sind die Ziele des Angreifers?}
	\begin{itemize}
		\item Abhören von Daten
		\item Modifizieren von Daten
		\item Maskerade und Erzeugen von Daten
	\end{itemize}
	\subsubsection{Was ist das "klassiche" Angeifermodell}
	Dolec\- Yao, Outsider. Er kann alles mithören, kann Dateneinheiten erzeugen und versenden und fremde Dateneinheiten modifizieren. Er kann jedoch nicht Enschlüsseln oder Verschlüsseln ohne den Schlüssel zu kennen.
	
	\subsection{Ansätze zum Schutz der Privatspähre im IoE?}
	Diensterbringung mit dem Prinzip der Datensparsamkeit, durch Samples mit ausschlieslich benötigten Daten möglichst Anonym.
	Anonymisierung durch herabsetzen der Präzesion auf ein Minimum, hinzufügen von Störwerten.
	Zentrale Datensenken sind zu vermeiden. 
	Identität der Quelle verschleidern, z.B. durch Pseudonyme
	Unverkettbarkeit von Samples gewährleisten
	
	\subsection{Was ist bei der Privatsphäre anders im IoE?}
	Die Technologie greift viel stärker ind private leben ein und erfasst dort Daten, dadurch ist die Privatsphäre stärker gefährdet als im klassischen Internet.
	
	\subsection{Was sind Ioe\- spezifische Heausforderungen für die Privatsphäre?}
	\begin{itemize}
		\item Heterogene Geräte: nicht genügen Ressourcen für klassische kryptografie.
		\item Häufig keine zentrale Infrastruktur nutzbar, kein Public Key, kein zentraler Vertrauensanker.
		\item unklares Vertrauensmodell: Gegen wen schützen? wer ist mein Vertrauensanker?
		\item Kein klassisches Angreifermodell
	\end{itemize}
	\subsubsection{Wo ist das Angreifermodell anders?}
	Im IoE sind die Geräte meist öffentlich zugänglich wodurch ein Angreifer physisch Zugang zu dem Gerät erhalten kann. Er kann den Speicher auslesen oder die Programmierung verändern.
	
\section{Recht}
	
	
						
