\section{Allgemeines}
	\subsection{Welche Themen wurden in der Vorlesung behandelt?}
		\begin{itemize}
			\item Klassifikation von Geräten
			\item Privatsphäre im IoE
			\item Betrachtung des IoE in Bezug auf das OSI-Modell
			\item Sicherheitsaspekte
		\end{itemize}
	
	\subsection{Auf welche Probleme stößt man im IoE? Warum müssen diese extra behandelt werden?}
		\begin{itemize}
			\item Möglichste Energiesparend, da Batteriebetrieben
			\item Sehr beschränkte Rechenleistung
			\item Kommunikationswege nicht zuverlässig, Semi-Broadcast
			\item Meist keine Infrastruktur vorhanden, wenig zentrale Infrastruktur
			\item Geringe Bauform und muss günstig sein
			\item Greif tief in die Privatsphäre des Menschen ein, Sammelt private Daten.
			\item Omnipräsenz?
		\end{itemize}
		
	\subsection{Welche Unterschiede und Besonderheiten gibt es?}
		\begin{itemize}
			\item Dezentral
			\item Selbstorganisieren
			\item Limitierte Ressourcen
			\item Unzuverlässiger Ressourcenkanal
			\item Unsicher, knoten können zerstört oder ausgelesen werden.
		\end{itemize}
		
	\subsubsection{Was muss man deswegen besonders beachten?}
		Durch die Allgegenwärtigkeit der Knoten muss besonders auf die Privatsphäre geachtet werden.
		
	\subsubsection{Privatsphäre was ist das?}
	Die Möglichkeit vertraulich zu kommunizieren und selbst darüber zu bestimmen, wer welche Daten über mich erhält.
	
	\subsubsection{Warum ist Privatsphäre wichtig? Wo ist das Problem beim Datenschutz?}
	sensitive Daten?
	
	\subsubsection{Was sind die Prinzipien des Datenschutzes?}
	Datensicherheit, Datensparsamkeit, Rechtmässigkeit, Transparenz, Nutzerrechte, Kontrolle.
		
	\subsubsection{Was gibt es für Schutzziele?}
	
	\subsubsection{Wie realisieren wir Sicherheit im IoE?}
	
	\subsubsection{Wir hatten eine Wetterstation als Beispiel, wie funktioniert das?}
	Wetterstation sammelt Daten und überträgt diese an die Server des Herstellers. Kunde greift mit seinem Smartphone auf die gesammelten Daten auf den Servern des Herstellers zu. Die Daten liegen nicht beim Kunden sondern beim Hersteller in fremder Hand.
	\paragraph{Wo liegt hier das Problem?}
		Es wird unter anderem die Lautsärke gemessen, die lässt eventuell sogar Sprachaufzeichnungen zu. Desweiteren können über diese Daten auf das Verhalten der Bewohner schließen lassen.
	\subsubsection{Wo liegt das Problem bei Smart Metering?}
	Aus dem hochfrequenten sampling der Verbrauchswerte lassen sich ebenso Verhaltensmuster schließen.
	\paragraph{Warum will man das dann?}
		Durch die hochfrequenten Messwerte lassen sich Geräte schalten wenn Stromüberschüsse vorhanden sind außerdem kann der Netzbetreiber mit diesen Informationen sein Stromnetz besser regeln und weiß wann er mehr oder weniger erzeugen muss.
	\paragraph{Womit lässt sich dieses Problem umgehen?}
		SMART-ER
	\subsection{Was ist das Problem bei vielen Geräten auf kleinem Raum, und was macht man dagegen?}
	Das Problem sind Kollisionen und der damit verbundene Energieverbrauch durch Erkennung und Retransmission. Das Routing wird komplexer. Die Gegenmaßnahme ist Topologiekontrolle.
	
	\subsection{Was hat bei Sensorknoten den höchsten Energieverbrauch?}
	Funkschnittstelle und, wenn vorhanden, Display.
	
	\subsection{Welche Netztopologien kommen im IoE vor?}
	\begin{itemize}
		\item Einzellnes Gateway
		\item Mehrere Gateways (Redundanz)
		\item Ad\- Hoc\- Netz
	\end{itemize}
		
\section{Routing}
	\subsection{Welche Verfahren gibt es?}
		Probalistische, lokalisationsbasierte und inhaltsbasierte Verfahren.
		
	\subsection{Welches inhaltsbasierte Verfahren hatten wir?}
		Direct Diffusion
	
	\subsubsection{Wie funktioniert Direct Diffusion?}		
		Sender broadcastet Interesse nach bestimmten Daten in Form von Attribut-Wert Paaren. System speichert Richtung aus der ein Intresse kam in Form von Gradienten. Sensoren mit entsprechenden Daten schicken die Daten an den Sender. Sobal der Sender merkt, dass die Daten verfügbar sind, startet er eine Reinforcment Phase zur Verstärkung der Gradienten.
		\subsubsection{Genaures im Detail } %TODO
		

\section{MAC}
	\subsection{Welche Verfahren hatten wir auf der MAC-Schicht?}
		S-MAC, B-MAC und 802.15.4 
		
		\subsubsection{Wie lassen sich diese Klassifizieren?}
		Zentral(802.15.4) Dezentral(S-Mac), Synchron(S-Mac) Asynchron(B-Mac) %TODO
		
	\subsection{Wie funktioniert S-MAC?}
	%TODO	
	
	\subsubsection{Welche Erweiterungen gab es bei S-MAC und wie funktionieren diese?}
		\begin{itemize}
			\item ALP:
			\item MP (Message Passing):
		\end{itemize}
		
	\subsection{Wie funktioniert B-MAC?}
	
	\subsection{Wie funktioniert 802.15.4 im Beacon Modus?}
	Duty Cycling, gibt das Verhältnis zwischen wach und schlafphasen an.
	\subsubsection{Wie erhält man im Beacon Modus einen garantierten Zeitschlitz?}
	%TODO Zeit Paket Diagram aus der Vorlesung zeichen
	
	


\section{Sicherheit}
	\subsection{Warum wollen wir vertraulich kommunizieren}
	
	\subsection{Reicht Verschlüsselung um die Privatsphäre zu schützen}
	Nein, Beispiel SMART\- METER
	\subsection{Was haben wir besprochen?}
	Single Mission Key, Zufallsverteilte Schlüssellisten und Key Infection
	
	\subsection{Wie funktioniert Single Mission Key?}
	Ein Schlüssel für alle Systeme, alle nutzen diesen für Verschlüsselung und Integritätssicherung. Problem ist, dass ein kompromitiertes Gerät reicht umd die gesamte Sicherheit zu brechen.
	\subsubsection{passiert wenn ein neues System hinzukommt?}
	Es wird mit dem selben Single Mission Key konfiguriert und kann dann mit dem Rest des Netzes kommunizieren.
	
	\subsection{Was ist EGLI?}
	Eschenauer Gligor
	\subsubsection{Wei funktioniert EGLI?}
	Es gibt eine globale Schlüsselliste und lokale zufällige Teilmengen als Keyrings.
	
	%TODO Digagramm des Schlüsselaustauschs
	A sendet Zufallszahl, B antwortet mit verschlüsselter Zahl mit allen Schlüsseln, damit kennt A nun die Schlüssel(durch Entschlüsseln mit eigenen Schlüsseln) 
	
	Keypool P wird vom Benutzer zufällig erzeugt. Jedes System erhält einen Keyring R mit einer Teilmenge von P. Es wird gezeigt, dass mit guter Wahl von P und R eine Wahrscheinlichkeit von mehr als 99\% erreicht werden kann, dass zwei Systeme einen gemeinsamen Schlüssel in ihren Keyring haben. 
	\paragraph{Ist EGLI zuverlässig?}
	Nein, da es keine E2E Quittungen gibt.
	\subsubsection{Wie Kann man den Aufwand minimieren?}
	Wert selbst mitschicken und damit nur mit richtigem entschlüsseltem Antworten.
	
	\subsubsection{Was ist Zuverlässigkeit?}
	Vollständige Übertragung in der korrekten Reihenfolge ohne Dupplikate.
	\paragraph{Wie stellen wir Zuverlässigkeit sicher?}
		Time, Quittungen, Sequenznummern, FEC, E2E Quittungen, Hop by Hop Quittungen.
	\subsubsection{Was ist der Unterschied zwischen E2E und Hop 2 Hop Kommunikation?}
	
	\subsection{Was ist Privatsphäre?}
	Right to be let alone, hat sich zu Recht auf Bestimmung über Verwendung persönlicher Daten entwickelt.
	\subsubsection{Beispiele für Geräte die persönliche Daten aufzeichnen?}
	Smart Watch, SmartPhone, Thermometer, Smart Meter...
	\subsubsection{Was ist bei Stromzählern problematisch? Die werden doch heutzutage auch schon abgelesen?}
	Die zeitliche Auflösung ist das Problem, bisher gab es einen Verbrauchswert pro Jahr, jeder sind Werte im Bereich von weniger als einer Sekunde möglich. Dies lässt Rückschlüsse auf das private Leben schließen.
	\subsubsection{Warum benötigt man Smart\- Meter?}
	Dezentrale Steuerung des Stromnetzes v.a. bei volatilen regenerativen Energiequellen (dezentral Energiequellen oder Verbraucher passend ein\-  und ausschalten, z.B. schaltet sich die Waschmaschine erst bei lokalem Stromüberschuss an.)
	\subsubsection{Wie kann man Schutz der Privatsphäre bei Smart\- Metern absichern?} 
	\begin{itemize}
		\item Grundmaßnamen sind immer Pseudoanonymisierung und Verschleierung (zeitliche und örtliche Auflösung)
		\item Pseudoanonymisierung ist jedoch problematisch da hier eine Trusted Third Party notwendig ist
		\item Zeitliche Verschleierung verhindert jedoch die effiziente Nutzung der Daten
		\item Besser ist hier eine örtliche Verschleierung durch Gruppenbildung (SMART\- ER)
	\paragraph{Wie funktioniert SMART\- ER?}
	%TODO Diagramm einfügen
	Gruppenbildung und Zufallszahlen austauschen
	
	Summe und Menge der Kommunikationspartner zurückgeben( weil Gültigkeit der Werte davon abhängt, ob Kommunikationspartner gültige Werte geliefert haben)
	
	Das Problem ist die Gruppeneinteilung durch Anbieter oder Trusted Third Party, so könnte ein Anbieter einfach jeden Zähler in eine eigene Gruppe zuweisen und das ganze Verfahren wäre sinnlos. Alternativen sind Smart Meter Speed\- dating(aber: Woher bekommen die Knoten vertrauenswürdig die gloable Liste aller Teilnehmer?) oder dezentrales Aggegieren in Overlay\- Netzen.
	\end{itemize}
	
	\subsection{Was ist das besondere im Bezug auf traditionelle Sicherheit?}
	\begin{itemize}
		\item Hauptproblem: Ressourcenknappheit, d.h. asymmetrische Verfahren sind problematisch, Schlüsselaustausch muss mit symmetrischen Verfahren implementiert werden.
		\item Oft gibt es keine zentrale Infrastruktur
		\item Schlüsselaustauschverfahren für klassische Sensornetzen: EGLI, Key Infection: DICE als schlankeres DTLS
	\end{itemize}
	
	\subsection{Wie funktioniert Key Infection?}
	%TODO Diagramm des Schlüsselaustausches zeichnen
	A sendet $k_i$, B sendet $Enc_{k_i}(k)$ zurück, dies ist nun der Sitzungsschlüssel.
	Nur wer in direkter Reichweite von A war, konnte $k_i$ abhören und kennt damit den Sitzungsschlüssel, damit muss der Angreifer sehr viele räumlich verteilte Systeme korrumpieren.
\section{noch nicht zugeordnet}
	\subsection{Wie kriegt man ein Sensornetz ans Internet angebunden?}
		Mit 6LoWPAN, angepasste Version von IPv6 für das IoT. Bildet Adaptionsschicht zwischen 802.15.4 und IPv6.
		\subsubsection{Was ist 6LoWPAN?}
		IPv6 für schwache Knoten, z.B. Sensornetzen. Ermöglicht Anbindung an das "normale" Internet. Bietet bspw. Header Compression. Ein Gateway setzt dann von 6LoWPAN in normales IPv6 um.
		\subsubsection{Wie funktioniert 6LoWPAN?}
		komprimierung/dekomprimierung am Edge-Router
		\subsubsection{Welche Arten von Adressierung gibt es in IPv6?}
		Unicast, Multicast und Anycast
		\subsubsection{Wie sieht eine 6LoWPAN Datenpaket aus?}
		Dispatch\- Byte (Header\- Typ/Fragmentierung), danach Paketheader
		Paketheader kann z.B. komprimierter IP\-  bzw. UDP\- Header sein, in diesem Fall enthält er ein Bitfeld, welches angibt welche Teile des Headers komprimiert werden.
		\subsubsection{Was ist das Dispatch/Compress Byte?}
		Das Dispatch Byte ist ein verpflichtendes "Headerfeld", das angibt ob es sich um einen komprimierten/unkomprimierten IPv6 Header handelt oder ob das Paket fragmentiert ist. Danach kommt nochmal ein Byte das anzeigt welches Feld des IP Heads komprimiert ist.
		\subsubsection{Welche Header Felder werden komprimiert und wie?}
		Direkt aus PAN bzw. MAC\- Adresse abgeleitet, konkateniert mit Netzpräfix, d.h. Präfix kann bei lokalen Adressen weggelassen werden, bei direkter Kommunikation kann die Adresse sogar ganz weg gelassen werden, das sie bereits im MAC\- Header steht.
		Alle Felder die bereits duch MAC Felder abgedeckt werden und ansonsten redundante Informationen darstellen würden.
		Verkehrsklasse/Flow Label, Version, Addressen, Länge...
		Versionsfeld wird komprimiert, da immer 6.
		
		Traffic-Control und Flow Label sind meist 0
		
		Adresse bei Link-Lokaler Kommunikation,  volle Adressen nur, wenn vorher ein Kontext etabliert wurde)
		
		Next Header
		
		Länge aus Schicht 2 
		
		\paragraph{Wie sieht ein komprimierter Header aus?}
		Mehrere Bits die angeben, ob Adresse, Länger, NextHeader etc... komprimiert oder unkomprimiert vorliegen, gefolgt von den entsprechenden Feldern.
		
		\subsubsection{In welche Richtung setzt das Gateway die komprimierten Header um?}
		Von IPv6 zu 6LoWPAN: komprimierung
		Von 6LowPan zu IPv6: dekomprimierung
		
		\subsubsection{Gibt es noch andere Header die komprimiert werden können?}
		Ja, UDP. Hierbei werden weniger Ports benutzt und die Länge auf Schicht 3 berechnet. Die Sequenznummer kann komprimiert werden, das UDP nicht zuverlässig ist.
		\paragraph{Wofür braucht man überhaupt Ports?} Bindung von Programmen an die Netzwerkadresse.
		\paragraph{Wieso benötigt UDP hier weniger Ports?} UDP ist kein zuverlässiges Protokoll und belegt deswegen keinen Port um auf eine Bestätigung des Empfangs zu hören.	Ausserdem kann die Anzahl der verfügbaren Ports reduziert werden da so ein ressourcenarmes Gerät vermutlich keine $2^{16}$ Anwendungen ausführen wird.	
		
		
		
		
		
		\subsubsection{Wie lang ist eine IPv6\- Adresse?}
		128 Bit. Die ersten 64Bit sind der Präfix, welche bei link-lokaler Kommunikation nicht benötigt wird und die zweiten 64 Bit sind die Interface ID, z.B. kann hier die MAC-Adresse des Adapters verwendet werden.
		\subsubsection{Wie lange ist ein IPv6\- Header?}
		40 Byte
		\subsubsection{Wie groß ist ein 6LoWPAN Paket?}
		\subsubsection{Wie groß ist ein 801.15.4 Frame?}
		
		\subsubsection{Wie funktioniert das Routing bei 6LoWPAN?}
		RPL???? DAG? NodeRank? Wurzel? RootRank??
		\subsubsection{Wie funktioniert Address Autoconfig?}
		\subsubsection{Wie funktioniert Fragmentierung?}
	
	\subsection{Welche Kommunikationsformen gibt es in WPANs?}
		Unicast, Multicast und Concast
		\subsubsection{Was ist Unicast?}
			\paragraph{Was ist HHR?}
			\paragraph{Was ist E2ER?}
		\subsubsection{Was ist Multicast?}
		\paragraph{Gibt es bei Multicast ein spezielles Verfahren bezüglich der Zuverlässigkeit, wenn ja welches und wie funktioniert es?}
			Pump Slowly Fetch Quickly: %TODO
			\subparagraph{Wie funktioniert das?}
			Szenario: Sender sendet, System leitet nach Verzögerung weiter
			\subparagraph{Leiten sie immer weiter?}
			Nein, wenn mindestens 3 andere bereits weitergeleitet haben, dann nicht
			\subparagraph{Und ansonten immer?}
			Neun, nur wenn das System alle Sequenznummern erhalten hat. Fehlet eine Sequenznummer, so wird diese erstmal mit einem NACK an alle Nachbarn angefragt. Nachbarn warten zufällige Zeit pro Sequenznummer und verschicken das Paket nur nochmal, falls nicht schon ein anderes System auf das NACK reagiert hat.
			\subparagraph{Wann und wie werden NACKs verbreitet?} 1Hop Umgebung (semi-bc)
			\subparagraph{Wann werden Daten gepumpt?} Probalistisches Multicast bis zu gewissen grenzen, kein Fluten
			\subparagraph{Wenn ein Knoten anhand der Sequenznummer feststellt, dass ihm ein Paket fehlt, muss er dann seine Nachbarn kennen um das NACK zu senden?}
			Nein, das NACK wird per Broadcast mit TTL=1 gesendet.
			\subparagraph{Haben wir damit jetzt 100\% zuverlässigkeit?}
			Nein, kommt keine Dateneinheit mit falscher Sequenznummer, so sieht auch kein Knoten die Notwendigkeit ein NACK zu senden.
			\subparagraph{Wie addressiert der Sender die Empfänger?}
			Wenn alle erreicht werden sollen, wird per Broadcast transmittet, soll nur eine Teilgruppe der Teilnehmer erreicht werden, so wird per Multicast addressiert %TODO ähm?
			\subparagraph{Kann ich bei einem Broadcast die Adresse einfach weglassen?}
			Nein man benutzt eine Broadcastadresse.
			
		\paragraph{Was ist ESRT und wie funktioniert es?}
		Senke broadcastet gewünschte Rate über starke Sendeleistung, da es meistens am Stromnetz hängt.
			
		
		\subsubsection{Was ist Concast?}
		Viele Sender senden zu einem Empfänger(Senke)
		\paragraph{Welches Protokoll haben wir hier erwähnt?}
			ESRT 
			\subparagraph{Was ist ESRT?}
			
			\subparagraph{Wie funktioniert ESRT?}
			
			\subparagraph{Wie wird die neue Senderate publiziert?}
			Broadcast von der Senke an alle Quellen. Dafür muss sie aber auch alle mit ihrer Senderstärke erreichen können.
			
		\subsubsection{Was ist ein Semi\- Broadcast Medium?}
		Funk, nur eine Teilmenge aller Empfänger liegt in der Sendereichweite
		
	\subsection{Wie funktioniert LEACH?}
	Zeit wird in Runden diskreditiert, in jeder Runde werden zufällig $N \times P$ Cluster\- Heads bestimmt.
	Es gibt jeweils $\frac{1}{P}$ Runden zusammengefasst, innerhalb dieser Runden wächst die Wahrscheinlichkeit, dass sich ein Knoden als Cluster\- Head definiert, bis auf 1 an. 
	
	Nach der Wahl der Cluster\- Heads: Andere Knoten wählen ein Cluster
	
	%TODO ich hab keine Ahnung
	
	\subsubsection{Wie wird in LEACH Energie gespart?}
	Nur die Cluster\- Heads müssen jederzeit mit Gateway o.ä. kommunizieren (hohe Sendeleistung).
	Die Cluster\- Heads stellen einen Zeitplan auf und die anderen Knoten kommunizieren immer nur mit den Cluster\- Heads (Sterntopologie) und müssen nur wach sein, wenn es der Zeitplan vorsieht.
	
	\subsection{Was ist CoAP}
	CoAP ist eine leichtgewichtige Variante von HTTP(kompaktes binäres Protokoll, dass auf UDP aufbaut).
	Anfragetypen, zwei Schichten, Pakettypen.
	
	\subsection{Mac-Protokolle? Was anstatt WLAN?}
	IEEE 802.15.4, B-MAC, S-MAC
	
\section{802.15.4}
	\subsection{Was ist 802.15.4}
	PHY + MAC Protokoll im IoE
	Soll WLAN/Bluetooth ersetzen.
	
\section{Geräteklassen}
	\subsection{Was sind Nanonetze?}
	Kleinste vernetzte Nanomaschinen die jeweils nur eine Aufgabe wie Speichern, Messen, Berechnen oder Manipulieren ausführen können. Nanonetze benutzen Molekulare Kommunikation anstatt elektromagnetische.
	
	\subsection{Was ist Smart Dust?}
	Sehr kleine beschänkte Hardware im Nano bis Millimeter Bereich. Die in kooperation zusammen wirken. Bisher nur Forschungsthematik. 
	
	\subsection{Was sind klassische Sensornetze?}
	Sehr viele kleine auf Einzelanwendungen maßgeschneiderte Systeme. Sie sind batteriebetrieben und haben nur geringe Hardwareressourcen.
	
	\subsection{Was ist Physical \& Embedded Computing}
	Flexible Systeme aus Standardhardware, welche kostengünstig ist und eine hohe Energieeffizient aufweist. Im gegensatz zu klassischen Sensornetzen können diese Geräte auch einen ständigen Stromanschluss haben.
	
	\subsection{Was ist Smart\- und Submetering}
	Die zeitnahe Erfassung (und Steuerung) von Energieverbäuchen (~15min). Soll die effizientere Nutzung von Ressourcen ermöglichen. Das deutsche Modell (BSI) sieht viele Zähler und einzellne Gateways vor. 
	
	\subsection{Was ist Smart Home?}
	Hausautomation und monitoring durch Sensornetze. 
	Z.B. zur Rolladen oder Licht Steuerung.
	Dabei kommen Probleme wie einfachheit vs. Sicherheit auf. 
	\subsubsection{Wie sieht es mit Smart Home und Privacy aus?}
	Die Sensoren sind tief in den Lebensraum verflochten und können höchst private Daten sammeln.
	\subsection{Wie sieht es mit Robotern und mobilen Plattformen aus?}
	Hier handelt es sich um eine mobile Plattform mit Sensoren udn Aktoren zur Messung vor Ort.
	Sie könenn in lebensfeindlichen oder unzugänglichen Umgebungen eingesetzt werden.
	\subsection{Was sind Wearables?}
	Am Körper Tragbare Sensoren die das persönliche Verhalten, Aktivitätsdaten und Vitalwerte erfassen.
	Die Auswertung und Anzeige ist auf diesen Geräten nur rudimentär möglich.
	\subsection{Was sind Smart Phones?}
	Sie stellen die heutige Schnittstelle zwischen Mensch und IoE da, sie erlauben die Steuerung von Remote-Geräten.
	Sie bestehen aus leistungsfähiger Hardware und setzen die Existenz von leistungsfähiger Kommunikationsinfrastruktur voraus.
	\subsection{Was sind Single\- Board Computer?}
	Kostengünstige Hardware für die Entwicklungs und Prototypen Umgebung.
	Sie haben eine große Anzahl an verschiedener Bus Systeme und Anschlussmöglichkeiten.
	
	\subsection{Was ist Industrie 4.0 \\ Industrial Internet?}
	Eine heterogene Umgebung an unterschiedlichsten Sensoren. Beinhaltet Leistungsfähige Steuer\- und Regelsysteme.
	Drahtlose Sensorik über Batterie oder Energy Harvesting.
	Anforderungen: Zuverlässigkeit, Robustheit, Langlebigkeit.
	
\section{Privatsphäre}
	\subsection{Was ist RFID und wo liegen die Herausforderungen für den Datenschutz?}
	RFID steht für Radio Frequency Identification. Drahtloste Übertragung von Informationen zwischen einem Transponder und einer Basistation.
	Das Problem ist das über diese Transponder eine eindutige Identifikation eines Objekts/Produkts oder einer Person möglich ist. Dieser Transponder kann auch aus großer Entfernung ausgelesen werden.
	\paragraph{Gibt es Lösungsansätze?}
	Ein möglicher Lösungsansatz wäre einen RFID Tag in einem Produkt an der Kasse duch einen "Kill" Befehl zu zerstören.
	Desweiteren könnten Kryptografische Mechanismen zur Authentifizierung genutzt werden.
	Physikalisch könnte auch einfach die Antenne abgerissen werden.
	
	\subsection{Wie sieht es mit dem Schutz der Privatsphäre aus?}
	Schutz des Persönlichkeitsrechts bei der Verarbeitung personenbezogener Daten 
	\subsubsection{Welche Daten sind schützenswert?}
	Laut BDSG sind alle Personenbezogenen Daten schützenswert allerdings sind im Kontext des Internet of Everything auch alle Metadaten schützenswert. Das heißt wer mit wem wann kommuniziert etc..
	
	\subsection{Was sind die Säulen zum Schutz der Privatsphäre?}
	\begin{itemize}
		\item Regulierung: Bundesdatenschutzgesetz etc..
		\item Selbstreuglierung: z.B. Gütesigel die den vertraulichen Umgang mit Daten bescheinigen.
		\item Selbstschutz: Privat Enhancing Technologies z.B. TOR
	\end{itemize}
	
	\subsection{Was sind allgemeine Schutzziele?}
	Anforderungen die Erfüllt werden müssen um schützenswürdige Güter vor Bedrohung zu schützen
	\begin{itemize}
		\item Confidentiality(Vertraulichkeit): Ermöglicht keine unautorisierte Informationsgewinnung
		\item Integrity(Integrität)
		\begin{itemize}
			\item Starke Integrität: Es ist unmöglich Daten unautorisiert zu verändern.
			\item Schwache Integrität: Es ist unmöglich Daten zu verändern ohne das es bemerkt wird.			
		\end{itemize}
		\item Avaliability(Verfügbarkeit): System beleibt Verfügbar und gewährt keine Einschränkung durch unautorisierten Zugriff.
	\end{itemize}

	\subsection{Spezielle Schutzziele für die Privatsphäre?}
	\begin{itemize}
		\item Unverkettbarkeit: Daten aus unterschiedlichen Kontexten sind nicht miteinander in Bezug zu setzen, z.B. durch Datenvermeidung oder Anonymisierung.
		\item Transparenz: Die Verarbeitung von Daten ist nachvollziehbar und überprüfbar. 
		\item Intervenierbarkeit: betroffene Personen können über die Erfassung und Verarbeitung ihrer Daten selbst bestimmen.
		
\end{itemize}		
	
	\subsection{Was ist ein Vertrauensmodell?}
	Vertrauen: Bewertung wie sich eine Entität in einem konkreten Sacherverhalt verhalten wird, hier die Annahme, dass Entität kein Angreifer ist.
	\begin{itemize}
		\item Vollständiges Vertrauen: uneingeschränktes Vertrauen aller Entitäten des Systems
		\item Keinerlei Vertrauen: Alle Entitäten potentielle Angreifer, PET notwendig
		\item Vertrauen in zentrale Instanz:Trusted third party
		\item Verteiltes Vertrauen: Nutzer vertraut lediglich, dass eine Teilmenge der beteiligten Entitäten nicht kooperiert.
		
	\end{itemize}
	\subsubsection{Was sind die Ziele des Angreifers?}
	\begin{itemize}
		\item Abhören von Daten
		\item Modifizieren von Daten
		\item Maskerade und Erzeugen von Daten
	\end{itemize}
	\subsubsection{Was ist das "klassiche" Angeifermodell}
	Dolec\- Yao, Outsider. Er kann alles mithören, kann Dateneinheiten erzeugen und versenden und fremde Dateneinheiten modifizieren. Er kann jedoch nicht Enschlüsseln oder Verschlüsseln ohne den Schlüssel zu kennen.
	
	\subsection{Ansätze zum Schutz der Privatspähre im IoE?}
	Diensterbringung mit dem Prinzip der Datensparsamkeit, durch Samples mit ausschlieslich benötigten Daten möglichst Anonym.
	Anonymisierung durch herabsetzen der Präzesion auf ein Minimum, hinzufügen von Störwerten.
	Zentrale Datensenken sind zu vermeiden. 
	Identität der Quelle verschleidern, z.B. durch Pseudonyme
	Unverkettbarkeit von Samples gewährleisten
	
	\subsection{Was ist bei der Privatsphäre anders im IoE?}
	Die Technologie greift viel stärker ind private leben ein und erfasst dort Daten, dadurch ist die Privatsphäre stärker gefährdet als im klassischen Internet.
	
	\subsection{Was sind Ioe\- spezifische Heausforderungen für die Privatsphäre?}
	\begin{itemize}
		\item Heterogene Geräte: nicht genügen Ressourcen für klassische kryptografie.
		\item Häufig keine zentrale Infrastruktur nutzbar, kein Public Key, kein zentraler Vertrauensanker.
		\item unklares Vertrauensmodell: Gegen wen schützen? wer ist mein Vertrauensanker?
		\item Kein klassisches Angreifermodell
		\item Es werden viel mehr Daten erfasst, zusammenführung beim Dienstanbieter?
		\item Kontinuierliche Datenerfassung
		\item Sensible Daten, aus dem persönlichen Umfeld/Wohnraum
		\item Veilfalt von gemessenen Phänomenen, die Verkettbarkeit erlaubt detailliere Profilbildung.
	\end{itemize}
	\subsubsection{Wo ist das Angreifermodell anders?}
	Im IoE sind die Geräte meist öffentlich zugänglich wodurch ein Angreifer physisch Zugang zu dem Gerät erhalten kann. Er kann den Speicher auslesen oder die Programmierung verändern.
	Hier wird von einem Insider Angriff ausgegangen.
	
	\subsection{Wie sieht es mit der Privatheit bei Smart\- Metering aus?}
	Das periodische Senden von Messwerten lieftert detailiertes Verbrauchsprofil und damit Einblicke in die Privatsphäre.
	
	\subsubsection{Was kann man dagegen tun?}
	Der einfachste Ansatz wäre Messdaten mittels "falscher" Identitäten oder Pseudonymen zu übertragen jedoch ist dies sehr aufwendig, so ist bei einem neuen Kunden eine komplette Neuvergabe der Pseudonyme notwendig da dieser sonst direkt zugeordnet werden könnte. Desweiteren wird eine Trusted-Third-Party zur Vergabe der Pseudonyme gebraucht, welche nicht vertrauenswürdiger sein muss als das Energieunternehmen.
	Außerdem kann das Profil des Pseudonyms über externe Daten zugeordnet werden, z.B. über Kenntnis von Arbeitszeit etc..
	
	Eine weitere Möglichkeit wäre es den Energiebedarf nach aussen über einen Energiespeicher zu maskieren. Dies funktioniert aber auch nur solange der Akku genügend Kapazität besitzt, zumal dieses Verfahren sehr kostspielig ist.

\subsubsection{Gibt es einen inteligenten Ansatz?}
	Ja, das Ziel von Smart\- Metering ist das aggregieren von Daten, daher wie hoch ist der Energieverbrauch in "Karlsruhe\- Mitte" oder von allen Kunden der EnBW. Von daher die Idee, die Daten zu Aggregieren bevor sie übermittelt werden. Dies kann z.B. über viele Haushalte oder einen langen Zeitraum geschehen. 
	Über einen langen Zeitraum ist einfach jedoch nicht Ziel von Smart\- Metering über viele Haushalte hinweg ist jedoch schwierig zu realisieren.
\subsubsection{Hatten wir dazu ein Verfahren?}
Ja wir hatten SMART\- ER, Smart Meterin Protocoll with Exactness and Robustness.
Das Grundkonzept ist die Einteilung in Gruppen welche durch den Messdienstleister durchgeführt wird. 
Innerhalb der Gruppen wird kooperiert so das die Messwerte maskiert aggregiert werden können.
Pro Messintervall
\begin{itemize}
	\item Austausch von Zufallswerten innerhalb der Gruppe 
	
	Hierzu werden die Ausgehenden Zufallszahlen von M abgezogen und die eingehenden auf M addiert. 
	\item Speichern der Kommunikationspartner (Abhängigkeiten)
	
	
	L beinhaltet alle Kommunikationspartner.
	\item Berechnen maskierter Messwerte
	
	Der Messwert wird maskiert indem er auf M addiert wird.
	\item Senden maskierter Messwerte und Abhängigkeiten an Messdienstleister
	
	L und M werden an den Dienstleister übermittelt.
	\item Eventuelle Bereinigung empfangene Messwerte durch Messdienstleister.
\end{itemize}

Hier bleibt das Problem bestehen das der Dienstleister die Gruppenbildung vornimmt. Sind nun alle bis auf den Eigenen Zähler einer Gruppe korrumpiert so kann genau auf die Werte des eigenen Zählers zurück geschlossen werden.

	\paragraph{Was kann man gegen das Problem bei der Gruppenbildung unternehmen?}
	Man muss die Gruppenbildung aus der Verantwortung des Messdienstleisters entziehen. Hier gibt es zwei Ansätze:	
	\begin{itemize}
		\item Smart Meter Speed Dating: Die Zähler bilden ihre Gruppen selbständig und dezentral. 
		\item Elderberry: Baumbasierter Ansatz mit strukturiertem P2P-Overlay. Dezentrale Aggregation.
	\end{itemize}
	
\section{Recht}
	\subsection{Was sind die Funktionen des Datenschutzrechts?}
	Das Datenschutzrecht dient dem Schutz der inneren und äußeren Freiheit der Persönlichkeitsentfaltung gegen Beeinträchtigung und Gefährdung bei der Verarbeitung personenbezogener Daten
	
	\subsection{Was wird vom Datenschutzrecht abgedeckt?}
	Der Umgang mit personenbezogenen Daten, Einzelangaben über persönliche oder sachliche Verhältnisse eines bestimmten oder bestimmbaren Betroffenen. 
	
	\subsection{Was ist vom Datenschutzrecht ausgenommen?}
	Datenverarbeitung für ausschließlich persönliche und familiäre Zwecke
	
	\subsection{Wie wird ein Datenverarbeitungsprozess aufgespaltet?}
	\begin{itemize}
		\item Datenerhebung
		\item Datenverarbeitung: Speicherung, Veränderung, Übermittlung...
		\item Datennutzung
	\end{itemize}
	
	\subsection{Wer ist Verantwortlich?}
	Datenschutzrechtlich Verantwortlich ist die Stelle, die über Zwecke und Mittel einer Datenumgangshandlung entscheidet.
	
	\subsection{Dürfen einfach so personenbezogene Daten erhoben werden?}
	Nein, der Erhebung muss zugestimmt werden
	
	\subsection{Dürfen die erhobenen Daten beliebig genutzt werden?}
	Bei der Zustimmung der Erhebung werden die Daten an einen bestimmten Zweck gebunden und dürfen nur für diesen benutzt werden.
	
\section{Kommunikation}

	\subsubsection{Was sind Cyber Physical Systems?}
	Cyber Physical Systems sind vernetzte Komponenten mit Sensoren und Aktoren die physikalische Prozesse steuern. So z.B. eine Heizungssteuerung oder die Steuerung einer Fabrikanlage.
	Sie werden als Bestandteil des Internet of Everything gesehen und haben teilweise hohe Anforderungen an Safety, somit z.B. geringe Latenzen oder setzten die richtige reihenfolge von Daten vorraus.
	
	\subsection{Medienzugriff}

	\subsubsection{Welche Probleme haben wir mit Funk als Medium?}
	Funk ist sehr unzuverlässig, es kommt zu hohen Fehlerraten. Da es sich um ein geteiltes Medium handelt muss der Zugriff darauf geregelt werden.
	
	\subsubsection{Welche Anfoderungen stellen wir an die drahtlose Kommunikation?}
	\begin{itemize}
		\item Hohe Lebenszeit der vernetzten Ding: niedriger Energieverbrauch
		\item Robusheit gegenüber Topologieveränderung, z.B. "schlafende" Systeme
		\item Skaliebarkeit
		\item Selbstorganisation
		\item Sicherheit und Schutz der Privatsphäre
\end{itemize}

	\subsubsection{Was ist ein Semi\- Broadcast\- Medium?}
	Drahtlose Netzwerke sind ein Semi\- Broadcast\- Medium da die Reichweite nicht unendlich ist und die Systeme nur die Dateneinheiten in ihrer Reichweite hören können, somit können Sender Kollisionen nicht erkennen denn sie treten beim Empfänger aus.
	
	Problem der "versteckten" und "ausgelieferten" Endsysteme
	
	\paragraph{Was sind verstecke Endsysteme?}
	Verstecke Endsysteme sind die Systeme die zwar mit dem Zielendsystem kommunizieren können nicht jedoch mit dem Quellsystem.
	
	\paragraph{Was sind ausgelieferte Endsysteme?}
	Ausgelieferte Endsysteme sind solche die im Bereich einer aktiven Kommunikation zwar im Sendebereich der Quelle aber nicht im Empfangsbereich des Ziels sitzen. Dieses ausgelieferte System will mit einem vierten System kommunizieren das sich nicht im Sendereich des Quellsystems befindet aber da sich das ausgelieferte System in diesem Bereich befindet ist für dieses System das Medium blockiert.  
	
	\subsubsection{Was ist das Problem bei nahen und fernen Endsystemen?}
	Ausgangslage sind drei Endsysteme. Endsystem A ist weit von Endsystem B und C entfernt, B und C jedoch nahe beieinander.
	Da die Signalstärke quadratisch zur Entfernung abnimmt kann es sein das wenn A und B gleichzeitig senden, A von B "übertönt" wird und C A nicht hören kann.
	
	\subsubsection{Was sind die grundlegenden Einordnungen der von und betrachteten Verfahren?}
	Es wurde Zeitmultiplexing und konkurrierende Verfahren betrachtet. Wobei besonders auf den Energieverbrauch und die Latenz der Kommunikation geachtet wurde.
	
	\subsubsection{Wieso achten wir auf den Energieverbrauch?}
	Sensorknoten sind häufig Batteriebetrieben, deswegen ist es eine wichtige aber sehr beschränkte Ressource.
	
	\subsubsection{Wie sparen wir Energie?}
	Indem wir den Funktransciever so häufig wie möglich deaktivieren
	
	\subsubsection{Was sind Beispiele für unnötigen Energieverbrauch?}
	\begin{itemize}
		\item Kollision: wenn mehrere Systeme gleichzeitig senden wird eine Sendewiederholung erforderlich
		\item Unnötiges Lauschen: Transciever ist aktiv obwohl nichts empfangen wird
		\item Mithören: System empfängt Dateneinheiten die gar nicht an es gerichtet ist.
	\end{itemize}
	\paragraph{Was kann man gegen unnötigen Energieverbrauch tun?}
	Gegen Kollisionen hilft Kollisionsvermeidung und gegen unnötiges Lauschen als auch Mithören hilft Duty-Cycling.
	
	\subsubsection{Wie funktioniert Kollisionsvermeidung?}
	Zur Kollisionsvermeidung wird eine seperate Signalisierung verwendet. Entweder über einen seperaten Kanal(Out\- of\- Band Signalisierung) oder über den gleichen Kanal(In\- Band\- Signalisierung). 
	Als Beispiel für die In\- Band\- Signalisierung: Multiple access with collison avoidance (MACA):
	\begin{itemize}
		\item Sender sendet kurze Request to Send (RTS) Dateneinheit
		\item Empfänger antwortet mit Clear to Send (CLS) Dateneinheit
		\item Sender sendet Daten
	\end{itemize}
	
	\paragraph{Können während des MACA keine Kollisionen auftreten?}
	Doch aber mit geringerer Wahrscheinlichkeit, da RTS und CTS sehr klein sind.
	
	\paragraph{Wie sieht das mit RTS/CTS und dem versteckten Endsystem aus?}
	A und C wollen zu B senden. A sendet als erstes RTS. B antwortet mit CTS (teilt die Dauer der Belegung mit). C empfängt CTS und weißt deswegen das es nicht senden darf.
	
	\paragraph{Wie sieht das mit dem ausgelieferten Endsystem in RTS/CTS aus?}
	B will zu A senden und C will mit D kommunizieren. C empfängt RTS von B muss aber dann nicht warten, da es kein CTS von A empfängt.
	
	\subsubsection{Was ist Duty\- Cycling?}
	Duty\- Cycling ist die Idee den Funktransciever so oft wie möglich zu deaktivieren um Energie zu sparen, was leider die Latenz erhöht.
	
	Bei Duty\- Cycling werden die Funktransciever bei Bedarf aktiviert um auf Aktivität zu prüfen oder zu Senden.
	Es wird zwischen synchronem und asynchronem Duty\- Cycling unterschieden.
	\paragraph{Was ist synchrones Duty\- Cycling?}
	Synchrones Duty\- Cycling bezeichnet das koordinierte aufwecken der Funktransciever nach einem vorgegeben Plan.
	\paragraph{Was ist asynchrones Duty\- Cycling?}
	Asynchrones Duty\- Cycling bezeichnet das aufwecken der Funktransciever ohne Koordination.
	
	\subsubsection{Was ist S\- MAC?}
	Sensor MAC ist ein Medienzugriffsprotokoll welches die Knoten zeitlich Synchronisiert.
	\paragraph{Wie funktioniert S\- MAC?} 
	Die Idee hinter S\- MAC ist das durch koordiniertes Schlafen idle listening verhindert wird, dazu erwachen die System gemeinsam an einem "rendezvous\- point", also eine zeitliche Synchronisation.
	Hierzu wird ein Rahmen fester Länge genutzt der in 2 Phasen aufgeteilt wird.
	\begin{enumerate}
		\item Listen Phase: In dieser Phase findet eine Synchronisation statt und sofern gewünscht wird der Datenaustausch angestoßen. Diese Phase ist nocheinmal in 2 Phasen unterteilt, wobei pro Zeitschlitz nur maximal eine Dateneinheit(SYNC, RTS, CTS) vorkommen darf.
		\begin{itemize}
			\item Sync\- Phase: Synchronisation durch SYNC Dateneinheit, welche die Zeitspanne bis zum Beginn der nächsten Sleep Phase beinhaltet.
			\item RTS/CTS Phase: In dieser Phase wir über einen zufällig gewählten Zeitschlitz über RTS ein Datenaustausch angestoßen, dannach folgt Carrier Sense + RTS/CTS 
		\end{itemize}
		\item Sleep Phase: Sollte in der Listne Phase ein Datenaustausch angestoßen worden sein, so entfällt diese Phase, ansonsten wird für die Dauer dieser Phase die Funkschnittstelle temporär deaktiviert.
	\end{enumerate}
	
	\paragraph{Wie sieht so ein S\- MAC Rahmen aus?}
	
	
	\begin{figure}[H]
	\includegraphics[scale=0.4]{internetOfEverything/smac-rahmen}
	\end{figure}
	
	\paragraph{Wie lange muss die Listenphase sein?}
	Die Listenphase muss lange genug sein um SYNC, RTS und CTS innerhalb einer geeigneter Zeitschlitze zu übertragen, dabei ergibt sich die Läng durch Parameter der MAC und PHY Schicht, z.B. der Datenrate etc. sie ist also nicht frei wählbar.
	
	\paragraph{Wie lange muss die Sleep Phase sein?}
	Die Sleep Phase ergibt sich aus dem Rest des S\- MAC Zeitrahmens. Sie ist also frei wählbar.
	
	\paragraph{Wie berechnet man den Duty Cycle?}
	$\text{Duty Cycle} = \frac{t_{Rahmen}}{t_{Listen}}$
	
	\paragraph{Wie berechnet sich die S\- MAC Rahmenlänge?}
	Die Rahmenlänge ist von Listen Phase und dem Duty Cycle abhängig. Daher ergibt sich $t_{Rahmen}=t_{Listen}+t_{Sleep}$
	
	\paragraph{Wozu dient die Synchronisation?}
	Die Synchronisation soll dem System den Anfang der nächsten Sleep Phase vermitteln
	
	\paragraph{Wie funktioniert die Synchronisation?}
	Die SYNC\- Dateneinheit enhält die verbleibende Zeit bis zum Beginn der nächsten Sleep Phase, dabei lernt das System den Zeitplan von Nachbarn durch regelmässigen Austausch von SYNC Dateneinheiten. 
	Falls es noch kein Nachbar bekannt ist wird ein eigener Zeitplan gewählt. Hierdurch entstehen zeitlich synchronisierte "Inseln" welche jedoch durch die SYNC Dateneinheiten den Zeitplan benachbarter Inseln lernen und so die Grenzen überbrücken können.
	
	\paragraph{Was passiert mit Systemen die an Grenzen liegen?}
	Sie empfangen und verteilen mehrere Zeitpläne, haben dementsprechend weniger Sleep Time und benötigen deswegen mehr Energie
	
	\paragraph{Gibt es Erweiterungen?}
	Ja, Message Passing und Adaptive Listening
	
	\paragraph{Was ist Message Passing?}
	Möchte man eine große Einheit von Nutzerdaten übertragen, so steigt die Bitfehlerwahrscheinlichkeit mit der Länge der Dateneinheit. Nun kann man die Nachricht als ganzes Übertragen wobei es mit hoher Wahrscheinlichkeit zu einem Fehler kommt. Eine weitere Möglichkeit wäre die große Einheit in kleiner Einheiten aufzuteilen. Die Bitfehlerwahrscheinlichkeit würde sinken aber es würde ein Overhead durch RTS/CTS entstehen.
	Hier setzt Message Passing an. Es fragmentiert die Einheit in mehrere Dateneinheiten die aber alle als Burst nach einem einzigen RTS/CTS Handshake übertragen werden und dazwischen einzellne bestätigt werden. So muss nur die defekte kleine Dateneinheit neu übermittelt werden und der Overhead durch RTS/CTS wird vermieden.
	
	\paragraph{Was ist Adaptive Listening?}
	Adaptive Listening geht das Problem an, dass pro Listen/Sleep Phase nur eine Dateneinheit weitergereicht werden kann und es deswegen in einem Multihop-Szenario zu einer höheren Verzögerung kommt. 
	Hierzu führt Adaptive Listening nach der Übertragung einer Dateneinheit eine zusätzliche Phase ein um einen neuen Datenaustausch zu initiieren, die Adaptive Listening Phase(ALP). Hierzu wacht ein nicht an einer Übertragung beteiliger Knoten nach der Übertragungszeit, welche im CTS angekündigt wurde wieder auf und hört auf ein neues RTS
	
	\paragraph{Wie kann man den durchschnittlichen Energiebedarf pro Byte berechnen?}
	Gesammtbedarch aller Systeme geteilt durch die Anzahl von der Senke empfangener Bytes.
	
	\paragraph{Wie kann man die durchschnittliche Ende zu Ende verzögerung berechnen?}
	Summe aller Ende zu Ende Verzögerungen geteilt durch die Anzahl der Dateneinheiten.
	
	\paragraph{Wie berechnet sich der Ende zu Ende Goodput?}
		Gesamzahl von der Senke empfangener Bytes geteilt durch die Zeitspanne zwischen Versenden der ersten Dateneinheit bis zum Empfang der letzten Dateneinheit an der Senke.
		
	\paragraph{Was sind die Nachteile von S\- MAC?}
	Geringer Durchsatz, hohe Latenz durch lange Wartezeiten auf nächsten freien Rahmen
	
	
	
	\subsubsection{Was ist B\- MAC?}
	Berkley Media Access Control ist ein Medienzugriffsprotokoll welches keine zeitliche Synchronisation der Knoten vorsieht. Es soll durch kollisionsvermeidung und effiziente Kanalnutzung bei hohen und niedrigen Datenraten einen Energieefizienten Betrieb ermöglichen. Dabei wird der Kanal periodisch geprüft anstatt auf eine zeitliche Synchronisation zu setzen. Dazu wird in zwei Zustände unterteilt
	\begin{itemize}
		\item Low Power Listening(LPL): Der Ruhezustand in dem das System die überwiegende Zeit befindet. Es wacht gelegentlich nur kurz auf um auf Daten auf dem Kanal zu überprüfen und bleibt dann Wach wenn etwas Empfangen wird, andernfalls geht es wieder schlafen.
		\item Clear Channel Assesment(CCA): Der Zustand in dem ein System etwas Senden möchte. Hierzu wird überprüft ob der Kanal frei ist, wenn frei dann wird erst eine Präambel gefolgt von den eigentlichen Daten übertragen.
	\end{itemize}
	
	\paragraph{Welchen Nachteil hat B\- MAC?}
	Das Problem der Versteckten Endgeräte wird nicht behandelt und es findet bei großen Nachrichten keine Fragmentierung statt.
	
						
