\chapter{Computergrafik}

Zusammenfassung der Vorlesung "`Computergrafik"' aus dem Wintersemester 2014.\footnote{\url{http://cg.ivd.kit.edu/lehre/ws2014/cg/index.php}}

\section{Farben, Bilder und Perzeption}

\subsection{Bilder, Darstellung und Framebuffer}

\subsubsection{Gammakorrektur}
\begin{itemize}
	\item Bilder sind oft zu hell oder zu dunkel
	\item Korrekturfunktion für das menschlichem Empfinden vpn physikalischen Helligkeitsunterschieden
	\item Ein idealer Monitor bildet einen Pixel-Wert \(n\) auf die Intensität \(I(n)\) ab
	\item Vergleichsweise wenige Stufen sind ausreichend
\end{itemize}


\subsection{Licht, Sehen und Wahrnehmung}

\subsubsection{Perzeption vs. Messung}
\begin{itemize}
	\item Das menschliche Auge kann die spektrale Zusammensetzung von Licht nicht erfassen
	\item Das Auge (und das Gehirn) macht eingeschränkte Messungen und passt sich den äußeren Umständen an
\end{itemize}

\subsubsection{Das Auge}
\begin{itemize}
	\item Zapfen: Für photopisches Sehen (Tagsehen) und trichromatisches Farbsehen. Blau (7\%), Grün (37\%), Rot (56\%)
	\item Stäbchen: Für skotopisches, monochromatisches Sehen (Nachtsehen); lichtempfindlicher, überall auf der Retina
\end{itemize}

\subsubsection{Farbmischungen}
\begin{itemize}
	\item \textbf{Additiv}
	\begin{itemize}
		\item RGB: \(C=rR+gG+bB\)
		\item Farbkombination durch Addition der Spektren
		\item Biologisch und technisch motiviert
		\item Anwendung: Monitore, Beamer
	\end{itemize}
	\item \textbf{Subtraktiv}
	\begin{itemize}
		\item CMY(K): Jede Primärfarbe absorbiert einen Teil des Spektrums; Key ist beim Druck meist Schwarz
		\item Dualer Farbraum zu RGB
		\item Farbkombination durch Multiplikation der Spektren
		\item Anwendung: Film, Farbstifte, Farbdrucker
	\end{itemize}
	\item \textbf{HSV}
	\begin{itemize}
		\item Farbton (Hue), Sättigung (Saturation) und Helligkeit (Value)
		\item Darstellung als Zylinder oder hexagonaler Kegel
		\item Anwendung: Benutzerschnittstellen, da intuitiv
	\end{itemize}
\end{itemize}

\subsubsection{Farbraum}
Die Menge der Farben, die mit einem bestimmten Modell beschrieben werden können.

\subsubsection{Graßmannsche Gesetze}
Jeder Farbeindruc kann mit drei Grundgrößen beschrieben werden.


\subsection{Rasterbilder}